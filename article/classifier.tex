\section{Classifier Implementation}
We lay out a preparatory step to improve brain registration by specifically extracting the brain volume from the MRI scans.
Our solution utilises a machine learning enabled brain tissue classifier, and the software implementation is formulated to integrate with the SAMRI Generic workflow \cite{ioanas_optimized_2019}, in order to ensure broader usability and reproducible benchmarking.
It creates a mask of the brain region using a classifier, which is then used to extract the region of interest.
Two classifiers were trained, one for scans acquired with RARE sequences yielding $T_2$-weighted contrast and one scans acquired with gradient-echo EPI sequences yielding either BOLD \cite{bold} and CBV \cite{cbv} contrasts (see \cref{subsec:Data Set}).
The assignment of “brain” and “not brain” annotation to each voxel in the scans is performed via a trained U-Net, a popular neural network for medical image segmentation.

\subsection{Workflow integration}
%% re-write SAMRI integration as a separate subsection “Workflow Integration”, and at the end of it, also highlight the advantages this has for software distribution (i.e. via NeuroGentoo), reproducibility (article RepSeP infrastructure), and benchmark accessibility.
The brain extraction nodes of the workflow return both the masked input and the binary mask.
The latter is used to constrain image similarity metric estimation on the relevant region of interest (ROI), while the extracted brain volume is used to prevent drifting of extracranial hyperintensities into the ROI.
The registration transformation is applied to the unmasked data to make the process minimally destructive.

\subsection{Training Data}

To improve general-purpose application, training examples need to be drawn from a usually unknown probability distribution, which is expected to be representative of the space of occurrences.
We set up an occurence space from which the data of interest is drawn, consisting of all the different mouse brain MRI data sets coming from multiple experiments, with their corresponding labels.
Based on an approximation of the occurrence space, the network builds a general model that enables it to extrapolate and produce sufficiently accurate predictions in new cases.
As a training dataset, we use scans which were preprocessed with the SAMRI \cite{noauthor_ibt-fmi/samri_2019} Generic workflow.
This data thus contains scans mapped onto a bregma-centered standard \cite{ioanas_optimized_2019} space derived from the Toronto Hospital for Sick Children Mouse Imaging Center brain template \cite{dsu}.
A template-based mask is available in the same reference space, and constitutes a ground truth estimation.
As registration in the absence of brain extraction is prone to imperfections, the mask does not always align perfectly with the brain region of every slice and some scans had to be removed manually.
Examples of slices from black listed volumes can be seen in \cref{blacklist_example}.

\section{Evaluation}

A challenge of QC with regard to spatial features is that a perfect mapping to the template is undefined.
Similarity metrics are ill-suited for QC because they are used internally by registration functions, whose mode of operation is based on maximizing them.
Extreme similarity score maximization is not a desired outcome.
Particularly if nonlinear transformations are employed, this may result in image distortion which should be penalized in QC.
Additionally, similarity metrics are not independent, so this issue cannot be circumvented by maximizing a subset of metrics and performing QC via the remainder.
To address this challenge we developed four alternative evaluation metrics: volume conservation, smoothness conservation, variance analysis, and functional analysis.
In order to mitigate possible differences arising from qualitative template features, we use these metrics in a multivariate analysis of both templates and workflows.


\subsection{Volume Conservation}

\begin{sansmath}
\py{pytex_subfigs(
        [
                {'script':'scripts/vc_violin.py', 'label':'vcv', 'conf':'article/1col.conf', 'options_pre':'{.48\\textwidth}',
			'caption':'Comparison across workflows and target templates, considering both BOLD and CBV functional contrasts.'
                        ,},
                {'script':'scripts/vcc_violin.py', 'label':'vccv','conf':'article/1col.conf', 'options_pre':'{.48\\textwidth}',
                        'caption':'Comparison across workflows and functional contrasts, considering only matching template-workflow combinations.'
                        ,},
                ],
        caption='\\textbf{Registration via the SAMRI Generic workflow and template optimally and reliably conserves scan volume --- unlike with the Legacy workflow and template.}
        Volume change relative to each original scan volume, with coloured patch widths estimating distribution density, solid lines indicating the sample mean,and dashed lines indicate the inner quartiles.
        ',
        label='fig:vc',
        )}
\end{sansmath}

Volume conservation is based on the assumption that the total volume of the scanned segment of the brain should remain roughly constant after preprocessing.
Beyond just size differences between the acquired data and the target template, a volume increase may indicate that the brain was stretched to fill in template brain space not covered by the scan, while a volume decrease might indicate that non-brain voxels were introduced into the template brain space.

In order to best analyze volume conservation, a Volume Conservation Factor (VCF) is computed for each processed scan, whereby volume conservation is highest for a VCF equal to 1.
For the current implementation we define brain volume as estimated by the 66\textsuperscript{th} voxel intensity percentile of the raw scan before any preprocessing.
The arbitrary unit equivalent of this percentile threshold is recorded for each scan and applied to all preprocessing workflow results for that particular scan, to obtain VCF esimates
 --- \cref{eq:vcf}, where $v$ is the voxel volume in the original space, $v^\prime$ the voxel volume in the transformed space, $n$ the number of voxels in the original space, $m$ the number of voxels in the transformed space, $s$ a voxel value sampled from the vector $S$ containing all values in the original data, and $s^\prime$ a voxel value sampled from the transformed data.

\begin{equation} \label{eq:vcf}
        V\!C\!F
        = \frac{v^\prime\sum_{i=1}^m [s^\prime_i \geq P_{66}(S)]}{v\sum_{i=1}^n [s_i \geq P_{66}(S)]}
        = \frac{v^\prime\sum_{i=1}^m [s^\prime_i \geq P_{66}(S)]}{v \lceil0.66n\rceil}
\end{equation}

As seen in \cref{fig:vcv}, we note that VCF is sensitive to
the processing workflow (\py{boilerplate.fstatistic('Processing', condensed=True)}),
the template (\py{boilerplate.fstatistic('Template', condensed=True)}),
but not the interaction thereof (\py{boilerplate.fstatistic('Processing:Template', condensed=True)}).

The performance of the Generic SAMRI workflow (with the Generic template) is significantly different from that of the Legacy workflow (with the Legacy template), yielding a two-tailed p-value of \py{pytex_printonly('scripts/vc_t.py')}.
Additionally, the root mean squared error ratio strongly favours the Generic workflow
($\mathrm{RMSE_{L}/RMSE_{G}\simeq} \py{pytex_printonly('scripts/vc_rmser.py')}$).

Descriptively, we observe that the Legacy level of the template variable introduces a notable volume loss
(VCF of \py{boilerplate.factorci('Template[T.Legacy]')}),
while the Legacy level of the preprocessing variable introduces a volume gain
(VCF of \py{boilerplate.factorci('Processing[T.Legacy]')}).
Further, we note that there is a very strong variance increase in all conditions for the Legacy processing workflow
(\py{boilerplate.varianceratio(template='Legacy')}-fold given the Legacy template, and \py{boilerplate.varianceratio(template='Generic')}-fold given the Generic template).

With respect to the data break-up by contrast (\cref{fig:vccv}), we see no notable main effect for the contrast variable
(VCF of \py{boilerplate.corecomparison_factorci('Contrast[T.CBV]')}).
We do, however, report a notable effect for the contrast-template interaction, with the Legacy workflow and CBV contrast interaction level introducing a volume loss
(VCF of \py{boilerplate.corecomparison_factorci('Processing[T.Legacy]:Contrast[T.CBV]')}).


\subsection{Smoothness Conservation}

\begin{sansmath}
\py{pytex_subfigs(
        [
                {'script':'scripts/scf_violin_templates.py', 'label':'scv', 'conf':'article/1col.conf', 'options_pre':'{.48\\textwidth}',
                        'caption':'Comparison across workflows and target templates, considering both BOLD and CBV functional contrasts.'
                        ,},
                {'script':'scripts/scf_violin_contrasts.py', 'label':'sccv','conf':'article/1col.conf', 'options_pre':'{.48\\textwidth}',
                        'caption':'Comparison across workflows and functional contrasts, considering only matching template-workflow combinations.'
                        ,},
                ],
        caption='
                \\textbf{Registration via SAMRI generic workflow and template conserves smoothness, enabling analysis with higher granularity, while the legacy workflow increases smoothing}
                Smoothness change relative to each original scan volume, with coloured patch widths estimating distribution density, solid lines indicating the sample mean,and dashed lines indicate the inner quartiles.
                ',
        label='fig:sc',
        )}
\end{sansmath}

A furher aspect of preprocessing quality is the resulting image smoothness.
Although controlled smoothing is a valuable preprocessing tool, which can be used to increase the signal-to-noise ratio (SNR), uncontrolled smoothness limits operator discretion for the trade-off between signal-to-noise, and feature granularity.
Uncontrolled smoothness can thus lead to undocumented and implicit loss of spatial resolution and is therefore associated with worse anatomical alignment in images and statistical maps \cite{fmriprep}.
We employ --- analogously to the Volume Conservation Factor --- a Smoothness Conservation Factor (SCF), which normalizes the smoothness of the preprocessed images with respect to the smoothness of the original images.
Our smoothness measure is the full-width at half-maximum (FWHM) of the signal intensity spatial autocorrelation function (ACF \cite{eklund2016cluster}).
Since fMRI data usually does not have a gaussian-shaped spatial ACF, we use AFNI \cite{cox1996afni} to fit the following function in order to compute the FWHM ---
\cref{eq:acf}, where $r$ is the distance of two noise samples, $a$ controls the weight of the gaussian term, $b$ the width of the gaussian and $c$ the decay of the mono-exponential term \cite{cox2017fmri}.

\begin{equation} \label{eq:acf}
        ACF(r)
        = a * e^{ -r^{2}/ (2 * b^{2}) } + (1 - a) + e^{-r/c}
\end{equation}

With respect to the data shown in \cref{fig:scv}, we note that SCF is sensitive to
the template (\py{boilerplate.fstatistic('Template', condensed=True, df_path='data/smoothness.csv', dependent_variable='Smoothness Conservation Factor')}),
the processing workflow (\py{boilerplate.fstatistic('Processing', condensed=True, df_path='data/smoothness.csv', dependent_variable='Smoothness Conservation Factor')}),
and the interaction of the factors (\py{boilerplate.fstatistic('Processing:Template', condensed=True, df_path='data/smoothness.csv', dependent_variable='Smoothness Conservation Factor')}).

The performance of the Generic SAMRI workflow (with the Generic template) is significantly different from that of the Legacy workflow (with the Legacy template), yielding a two-tailed p-value of \py{pytex_printonly('scripts/scf_t.py')}.
In this comparison, the root mean squared error ratio favours the Generic workflow
($\mathrm{RMSE_{L}/RMSE_{G}\simeq} \py{pytex_printonly('scripts/scf_rmser.py')}$).

Descriptively, we observe that the Legacy level of the template variable introduces a smoothness reduction
(SCF of \py{boilerplate.factorci('Template[T.Legacy]', df_path='data/smoothness.csv', dependent_variable='Smoothness Conservation Factor')}),
while the Legacy level of the preprocessing variable introduces a smoothness gain
(SCF of \py{boilerplate.factorci('Processing[T.Legacy]', df_path='data/smoothness.csv', dependent_variable='Smoothness Conservation Factor')}).
Further, we note that there is a variance increase for the Legacy processing workflow
(\py{boilerplate.varianceratio(df_path='data/smoothness.csv',dependent_variable='Smoothness Conservation Factor', max_len=3, template='Legacy')}-fold given the Legacy template and \py{boilerplate.varianceratio(df_path='data/smoothness.csv',dependent_variable='Smoothness Conservation Factor', template='Generic', max_len=3)}-fold given the Generic template).

Given the break-up by contrast shown in \cref{fig:sccv}, we see no effect for the contrast variable
(SCF of \py{boilerplate.corecomparison_factorci('Contrast[T.CBV]', df_path='data/smoothness.csv', dependent_variable='Smoothness Conservation Factor')}).
We do, however, report an effect for the contrast-template interaction, with Legacy workflow and CBV contrast interaction level introducing an increase in smoothness
(SCF of \py{boilerplate.corecomparison_factorci('Processing[T.Legacy]:Contrast[T.CBV]', df_path='data/smoothness.csv', dependent_variable='Smoothness Conservation Factor')}).

\subsection{Variance analysis}

\begin{sansmath}
\py{pytex_fig('scripts/variance_catplot.py',
        conf='article/varplot.conf',
        label='var',
        caption='
                \\textbf{The SAMRI Generic workflow conserves subject-wise variability and minimizes trial-to-trial variability compared to the Legacy workflow.}
                Swarmplots illustrate similarity metric scores of preprocessed images with respect to the corresponding workflow template, plotted across subjects (separated into x-axis bins) and sessions (individual points in each x-axis bin).
                ',
        multicol=True,
        )}
\end{sansmath}

An additional way to assess preprocessing quality focuses on the robustness to variability resulting from repeat experimentation, and whether this is attained without overfitting (i.e. compromising physiologically meaningful variability).
The core assumption of this analysis of variance is that adult mouse brains in the absence of intervention retain size, shape, and implant position during the 8 week study period.
Consequently, when examining similarity scores of preprocessed scans with respect to the target template, more variation should be found across levels of the subject variable rather than session variable.
This comparison can be performed using a type 3 ANOVA, modelling both the subject and the session variables.
For this assessment we select three metrics with maximal sensitivity to different features:
Neighborhood Cross Correlation (CC, sensitive to localized correlation),
Global Correlation (GC, sensitive to whole-image correlation),
and Mutual Information (MI, sensitive to whole-image information similarity independently of a correlation assumption).
%As similarity metrics are not equivalent, and since the main comparison of variance is performed across all of them, p-values for the variable effects are uncorrected.

\Cref{fig:var} renders the similarity metric scores for both the SAMRI Generic and Legacy workflows (considering only the matching workflow-template combinations).
For all three metrics, using the Legacy workflow, that the F-statistic is consistently higher for the session than for the subject variable:
CC (subject: \py{boilerplate.variance_test('C(Subject)','Legacy','CC', condensed=True)}, session: \py{boilerplate.variance_test('C(Session)','Legacy','CC', condensed=True)}),
GC (subject: \py{boilerplate.variance_test('C(Subject)','Legacy','GC', condensed=True)}, session: \py{boilerplate.variance_test('C(Session)','Legacy','GC', condensed=True)}),
and MI (subject: \py{boilerplate.variance_test('C(Subject)','Legacy','MI', condensed=True)}, session: \py{boilerplate.variance_test('C(Session)','Legacy','MI', condensed=True)}).
Notably, for the MI metric the effect of the session variable is significant, but not that of the subject variable. 

%Looking at the Generic SAMRI workflow we find the opposite effect.
%Across metrics, F-statistics for the subject variable are consistently higher than for the session variable:
%CC (subject: \py{boilerplate.variance_test('C(Subject)','Generic','CC', condensed=True)}, session: \py{boilerplate.variance_test('C(Session)','Generic','CC', condensed=True)}),
%GC (subject: \py{boilerplate.variance_test('C(Subject)','Generic','GC', condensed=True)}, session: \py{boilerplate.variance_test('C(Session)','Generic','GC', condensed=True)}),
%and MI (subject: \py{boilerplate.variance_test('C(Subject)','Generic','MI', condensed=True)}, session: \py{boilerplate.variance_test('C(Session)','Generic','MI', condensed=True)}).
%Notably, in the CC metric, the subject-wise variance reaches significance, but not the session-wise variance.

The Generic SAMRI workflow shows a reversing trend.
F-statistics for the subject variable are consistently higher than for the session variable:
CC (subject: \py{boilerplate.variance_test('C(Subject)','Generic','CC', condensed=True)}, session: \py{boilerplate.variance_test('C(Session)','Generic','CC', condensed=True)}),
GC (subject: \py{boilerplate.variance_test('C(Subject)','Generic','GC', condensed=True)}, session: \py{boilerplate.variance_test('C(Session)','Generic','GC', condensed=True)}),
and MI (subject: \py{boilerplate.variance_test('C(Subject)','Generic','MI', condensed=True)}, session: \py{boilerplate.variance_test('C(Session)','Generic','MI', condensed=True)}).


\subsection{Functional Analysis}

Functional analysis is a frequently used avenue for preprocessing QC.
Its viability derives from the fact that the metric being maximized in the registration process is not the same output metric as that used for QC.
This method is however primarily suited to examine workflow effects in light of higher-level applications, and less suited for wide-spread QC (as it is computationally intensive and only applicable to stimulus-evoked functional data).

As a first step we examine statistical power via the negative logarithm of first-level p-value maps (i.e. voxelwise statistical estimates for the probability that each voxel time course is --- by chance alone --- at least as well correlated with the stimulation regressor as the voxel time course measured).
We compute the per-scan average of these values, which we term Mean Significance (MS) --- \cref{eq:ms}, where $n$ represents the number of statistical estimates in the scan, and $p$ is a p-value.

\begin{equation} \label{eq:ms}
        M\!S = \frac{\sum_{i=1}^n -log(p_i)}{n}
\end{equation}

\begin{sansmath}
\py{pytex_subfigs(
        [
                {'script':'scripts/ms_violin.py', 'label':'msv', 'conf':'article/1col.conf', 'options_pre':'{.48\\textwidth}',
			'caption':'Comparison across workflows and target templates, considering both BOLD and CBV functional contrasts.'
                        ,},
                {'script':'scripts/msc_violin.py', 'label':'mscv','conf':'article/1col.conf', 'options_pre':'{.48\\textwidth}',
                        'caption':'Comparison across workflows and functional contrasts, considering only matching template-workflow combinations.'
                        ,},
                ],
        caption='\\textbf{Per-scan significance from functional analysis is not strongly susceptible to the preprocessing variants examined:.}
                The coloured patch width estimates distribution density, while continuous markers indicate the sample mean and dashed markers indicate the inner quartiles.',
        label='fig:ms',)}
\end{sansmath}

As seen in \cref{fig:ms}, MS is not sensitive to
the processing workflow
(\py{boilerplate.fstatistic('Processing', dependent_variable='Mean Significance', df_path='data/functional_significance.csv', condensed=True)}),
but is sensitive to the template
(\py{boilerplate.fstatistic('Template', dependent_variable='Mean Significance', df_path='data/functional_significance.csv', condensed=True)}),
and the interaction of both factors
(\py{boilerplate.fstatistic('Processing:Template', dependent_variable='Mean Significance', df_path='data/functional_significance.csv', condensed=True)}).

The SAMRI Generic workflow (with the Generic template) does not significantly differ from the Legacy workflow (with the Legacy template) in terms of MS, yielding a two-tailed p-value of \py{pytex_printonly('scripts/ms_t.py')}.

Descriptively, we observe that the Legacy level of the template variable introduces a notable significance increase
(MS of \py{boilerplate.factorci('Template[T.Legacy]', df_path='data/functional_significance.csv', dependent_variable='Mean Significance')}),
while the Legacy level of the preprocessing variable introduces no significant change
(MS of \py{boilerplate.factorci('Processing[T.Legacy]', df_path='data/functional_significance.csv', dependent_variable='Mean Significance')}),
and the interaction of the Legacy template and Legacy processing introduces a significance loss
(MS of \py{boilerplate.factorci('Processing[T.Legacy]:Template[T.Legacy]', df_path='data/functional_significance.csv', dependent_variable='Mean Significance')}),
Furthermore, we again note a variance increase in all conditions for the Legacy processing workflow
(\py{boilerplate.varianceratio(df_path='data/functional_significance.csv', template='Legacy', dependent_variable='Mean Significance')}-fold
given the Legacy template, and
\py{boilerplate.varianceratio(df_path='data/functional_significance.csv', template='Generic', dependent_variable='Mean Significance')}-fold
given the Generic template).

With respect to the data break-up by contrast (\cref{fig:mscv}), we see no notable main effect for the contrast variable
(MS of \py{boilerplate.corecomparison_factorci('Contrast[T.CBV]', df_path='data/functional_significance.csv', dependent_variable='Mean Significance')})
and no notable effect for the contrast-template interaction
(MS of \py{boilerplate.corecomparison_factorci('Processing[T.Legacy]:Contrast[T.CBV]', df_path='data/functional_significance.csv', dependent_variable='Mean Significance')}).

Overall statistical power is, however, independent of the mapping accuracy, and functional analysis effects can further be inspected by visualizing the statistic maps.
For a succinct overview capturing both amplitude and directionality of the signal, we present second-level t-statistic maps depicting the CBV and BOLD omnibus contrasts (across all subjects and sessions) in \cref{fig:m}.
%While the most salient feature of this figure is the qualitative distribution difference between CBV and BOLD scans, we note that this is to be expected given the different nature of the processes, and is wholly orthogonal to the question of registration.
Crucial to the examination of registration quality and its effects on functional read-outs is the differential coverage.
We note that the Legacy workflow induces coverage overflow, extending to the cerebellum (\cref{fig:mlgc,fig:mllc,fig:mlgb,fig:mllb}), as well as to more rostral areas when used in conjunction with the Legacy template (\cref{fig:mllc,fig:mllb}).
Separately from the Legacy workflow, the Legacy template causes acquisition slice misalignment (\cref{fig:mglc,fig:mllc,fig:mglc,fig:mllb}).
Positive activation of the Raphe system, most clearly disambiguated from the surrounding tissue in the BOLD contrast, is notably displaced very far caudally by the joint effects of the Legacy workflow and the Legacy template (\cref{fig:mllb}).
We note that processing with the Generic template and workflow (\cref{fig:mggc,fig:mggb}), does not show issues with statistic coverage alignment and overflow.

\py{pytex_subfigs(
	[
		{'script':'scripts/map_generic_cbv.py', 'label':'mggc', 'conf':'article/map.conf', 'options_pre':'{.48\\textwidth}',
			'caption':'
                                Generic workflow with Generic template CBV map, showing correct slice orientation and coverage correctly bounded to the acquisition area.
                                '
			,},
		{'script':'scripts/map_generic_ambmc_cbv.py', 'label':'mglc','conf':'article/map.conf', 'options_pre':'{.48\\textwidth}',
			'caption':'
                                Generic workflow with Legacy template CBV map, showing incorrect slice orientation and coverage correctly bounded to the acquisition area.
                                '
			,},
		{'script':'scripts/map_legacy_dsurqec_cbv.py', 'label':'mlgc','conf':'article/map.conf', 'options_pre':'{.48\\textwidth}',
			'caption':'
                                Legacy workflow with Generic template CBV map, showing correct slice orientation and coverage caudally extending beyond the acquisition area.
                                '
			,},
		{'script':'scripts/map_legacy_cbv.py', 'label':'mllc','conf':'article/map.conf', 'options_pre':'{.48\\textwidth}',
			'caption':'
				Legacy workflow with Legacy template CBV map, showing incorrect slice orientation and coverage both caudally and rostrally extending beyond acquisition area.
				'
			,},
		{'script':'scripts/map_generic_bold.py', 'label':'mggb', 'conf':'article/map.conf', 'options_pre':'{.48\\textwidth}',
			'caption':'
                                Generic workflow with Generic template BOLD map, showing correct slice orientation and coverage correctly bounded to the acquisition area.
                                '
			,},
		{'script':'scripts/map_generic_ambmc_bold.py', 'label':'mglb','conf':'article/map.conf', 'options_pre':'{.48\\textwidth}',
			'caption':'
                                Generic workflow with Legacy template BOLD map, showing incorrect slice orientation and coverage correctly bounded to the acquisition area.
                                '
			,},
		{'script':'scripts/map_legacy_dsurqec_bold.py', 'label':'mlgb','conf':'article/map.conf', 'options_pre':'{.48\\textwidth}',
			'caption':'
                                Legacy workflow with Generic template BOLD map, showing correct slice orientation and coverage caudally extending beyond acquisition area.
                                '
			,},
		{'script':'scripts/map_legacy_bold.py', 'label':'mllb','conf':'article/map.conf', 'options_pre':'{.48\\textwidth}',
			'caption':'
				Legacy workflow with Legacy template BOLD map, showing incorrect slice orientation and coverage both caudally and rostrally extending beyond acquisition area.
			        '
                        ,},
		],
	caption='
                \\textbf{Legacy processing leads to a problematic overflow of the statistic maps into adjacent anatomical regions, leaking beyond the acquistion area.}
                SAMRI mitigates this effect as illustrated by multiplanar depictions of second-level omnibus statistic maps separately evaluating CBV and BOLD scans, and thresholded at $\mathrm{|t|\geq2}$.
                The acquisition area is bracketed in pink, and in comparing it to statistic coverage it is important to note that the latter is always underestimated, as the omnibus statistic contrast is only defined for voxels captured in \\textit{every} evaluated scan.
                ',
	label='fig:m',)}

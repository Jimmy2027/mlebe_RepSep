\section{Methods}

For the quality control of the workflows, a dataset with an effective size of 102 scans is used.
Data from 11 adult animals is included, with each animal scanned on up to 5 sessions (repeated at 14 day intervals).
Each session contains an anatomical scan and two functional scans --- with Blood-Oxygen Level Dependent (BOLD) \cite{Ogawa1990} and Cerebral Blood Volume (CBV) \cite{Marota1999} contrast, respectively (for a total of 68 functional scans).

Anatomical scans were acquired via a TurboRARE sequence, with a RARE factor of 8, an echo time (TE) of \SI{21}{\milli\second}, an inter-echo spacing of \SI{7}{\milli\second}, and a repetition time (TR) of \SI{2500}{\milli\second}, sampled at a sagittal resolution of $\mathrm{\Delta x(\nu)=\SI{166.7}{\micro\meter}}$, a horizontal resolution of $\mathrm{\Delta y(\phi)=\SI{75}{\micro\meter}}$, and a coronal resolution of $\mathrm{\Delta z(t)=\SI{650}{\micro\meter}}$ (slice thickness of \SI{500}{\micro\meter}).
The functional BOLD and CBV scans were acquired using an Echo Planar Imaging (EPI) sequence with a flip angle of \SI{60}{\degree} and with $\mathrm{TR/TE = \SI{1000}{\milli\second}/\SI{15}{\milli\second}}$ and $\mathrm{TR/TE = \SI{1000}{\milli\second}/\SI{5.5}{\milli\second}}$, respectively.
Functional scans were sampled at $\mathrm{\Delta x(\nu)=\SI{312.5}{\micro\meter}}$, $\mathrm{\Delta y(\phi)=\SI{281.25}{\micro\meter}}$, and $\mathrm{\Delta z(t)=\SI{650}{\micro\meter}}$ (slice thickness of \SI{500}{\micro\meter}).
All aforementioned scans were acquired with a Bruker PharmaScan system (\SI{7}{\tesla}, \SI{16}{\centi\meter} bore), and an in-house T/R coil.

The measured animals were fitted with an optic fiber implant ($\mathrm{l=\SI{3.2}{\milli\meter} \ d=\SI{400}{\micro\meter}}$) targeting the Dorsal Raphe (DR) nucleus in the brain stem.
The nucleus was rendered sensitive to optical stimulation by transgenic expression of Cre recombinase under the ePet promoter \cite{Scott2005} and viral injection of rAAVs delivering a plasmid with Cre-conditional expression of Channelrhodopsin and YFP ---
pAAV-EF1a-double floxed-hChR2(H134R)-EYFP-WPRE-HGHpA, a gift from Karl Deisseroth (\href{https://www.addgene.org/20298/}{Addgene plasmid \#20298}).

The DR was stimulated via an Omicron LuxX 488-60 laser (\SI{488}{\nano\meter}) tuned to \SI{30}{\milli\watt} at contact with the fiber implant, according to the protocol listed in \cref{tab:stim}.
The operation and stimulation procedure, as well as general picture of obtained activation is consistent with previous results \cite{Grandjean2019}, and is not further commented in this study.

\section{Methods}
It is of foremost importance that the complexity of MRI processing workflows is manageable to their prospective users, such as biologists with only cursory programming experience.
One should, however, also be mindful that workflow transparency and reproducibility (a prerequisite for usage in scientific data analysis) are not compromised for trivial features.
We thus follow  a set of design guidelines, stating that:
(1) each workflow is represented by a high-level function, whose parameters correspond to operator-understandable concepts (i.e. describe the operations performed, rather than the computational manner in which they are performed);
(2) workflow functions are highly parameterized but include sane defaults (so that users \textit{can} change their function to a significant extent without needing to edit the constituent code);
and (3) graphical or interactive interfaces are wholly avoided (as they impede reproducibility, encumber the dependency graph, and reduce the sustainability of the project).

\begin{figure*}[h!]
	\begin{subfigure}{.64\textwidth}
		\centering
		\includedot[width=\textwidth]{data/generic}
		\vspace{1.4em}
		\caption{
			“SAMRI Generic” (new) workflow, based on the \textcolor{mg}{\texttt{antsRegistration}} function, and making no use of empirical value thresholding, or destructive operations such as affine manipulation or data matrix reordering.
			The pipeline uses a higher-resolution structural scan intermediary for registration (note the two processing streams).
			The function used is highly parameterized, and both of its instances --- “s\niceus register” and “f\niceus register” --- are supplied in the workflow with defaults optimized for mouse brain registration).
			}
		\label{fig:wfgg}
	\end{subfigure}\hfill
	\begin{subfigure}{.34\textwidth}
		\centering
		\includedot[width=\textwidth]{data/legacy}
		\vspace{-1.9em}
		\caption{
			“SAMRI Legacy” workflow, based on the \textcolor{mg}{\texttt{antsIntroduction.sh}} function, performing destructive affine manipulations, and not making use of a structural intermediary.
			The function used has hard-coded parameters optimized for human brain registration.
			}
		\label{fig:wfgl}
	\end{subfigure}
	\caption{
		Directed acyclic graphs depicting the two alternate MRI registration workflows and their constituent processing steps.
		The package correspondence of each processing node is appended in parantheses to the node name.
		The “utility” indication corresponds to nodes based on Python functions specific to the workflow, distributed alongside it, and dynamically wrapped via Nipype.
		The “extra\niceus interfaces” indication corresponds to nodes using explicitly defined Nipype-style interfaces, which are specific to the workflow and distributed alongside it.
		}
	\label{fig:wfg}
\end{figure*}

The language of choice for workflow handling is Python, owing to its Free and Open Source (FOSS) nature, readability, wealth of available libraries, ease of package management, and its large and dynamic developer community.
While workflow functions are written in Python, we also provide automatically generated Command Line Interfaces (CLIs), for use directly with Bash.
These autogenerated CLIs ensure that features become available in Bash and Python synchronously, and workflows behave identically regardless of the language in which they are invoked.

\subsection{Technologies}

Internally, the workflow functions make use of the Nipype \cite{nipype} package, which provides high-level workflow management and execution features.
Via this package, functions provided by any other package can be encapsulated in a node (complete with error reporting and isolated re-execution support) and integrated into a directed workflow graph.
Paralellization can also be managed via a number of execution plugins, allowing excellent scalability.
Most importantly, Nipype can generate graph descriptor language (DOT) summaries as well as visual representations (e.g. \cref{fig:wfg}) of the workflows.
These summaries are suitable for operator inspection, graph theoretical analysis, and programmatic comparison between workflow variants.

Via Nipype we utilize basic MRI preprocessing functions from the FSL package \cite{fsl} and registration functions from the ANTs package \cite{ants}.
While there is theoretically no limit to the number of external packages usable with Nipype, we constrain our choice as much as possible in order to minimize the dependency graph.
The choice of the ANTs package (in addition to FSL, which also provides registration functions) owes to the package providing registration functions which are more highly parameterized.
This feature allows us to avoid maladaptive (for our specific use case) optimization choices, and instead adapt the registration to the brain images at hand.
Additionally we use a number of functions specifically developed for our workflows, which aid in more case-specific tasks, such as BIDS \cite{bids} input, and dummy scans management.

Given the aforementioned guiding principles, and the hitherto listed technologies, we have constructed two registration workflows: The “Legacy” workflow (\cref{fig:wfgl}), which exhibits the common practices detailed in the \nameref{sec:bg}~section; and our novel “Generic” workflow.

For Quality Control we distribute as part of this publication workflows using the NumPy \cite{numpy}, SciPy \cite{scipy}, and pandas \cite{pandas} packages, as well as Statsmodels \cite{statsmodels} for top-level statistics and matplotlib \cite{matplotlib} and Seaborn \cite{seaborn} for plotting.
Specifically, distribution densities for plots are drawn using the Scott bandwidth density estimator \cite{Scott1979}.

% Markus Marks !!! add short technology paragraph for optimization methods.

\subsection{Distribution}

As registration is a crucial step of a larger data analysis process (rather than an analysis process in its own right), the workflows are best distributed as part of a full stack (i.e. from raw data to statistical summaries) workflow package.
We include the aforementioned Generic and Legacy workflows in the SAMRI (Small Animal Magnetic Resonance Imaging) data analysis package \cite{samri} of the ETH/UZH Institute for Biomedical Engineering.
This Free and Open Source package handles data analysis for numerous collaborators, and applies the guiding principles relevant to the registration workflows throughout all processing steps.

\subsection{Template Package}

\py{pytex_subfigs(
	[
		{'script':'scripts/dsurqec.py', 'label':'dsu', 'conf':'article/template.conf', 'options_pre':'{.48\\textwidth}',
			'caption':'The “Generic” template, which exemplifies $T_2$ contrast, a canonical MR \textit{and} stereotactc data matrix orientation, a standard header with an RAS orientation, and a realistic affine transformation.
				Note the origin at Bregma which provides histologically meaningful coordinates.'
			,},
		{'script':'scripts/ambmc.py', 'label':'amb','conf':'article/template.conf', 'options_pre':'{.48\\textwidth}',
			'caption':'The “Legacy” template, which exemplifies histological contrast, the canonical histological template data matrix orientation (shared e.g. by the Allen Brain Institute template), alongside a non-standard header with features such as an RPS orientation and inflated affine transformation.'
			,},
		],
	caption='Multiplanar depcitions of the mouse brain templates, with slice coordinates centered at zero on all axes.',
	label='fig:t',)}

The suitability of a registration workflow as a standard is contingent on the quality of the template being used.
Particularly size and orientation of the template may pose constraints on the workflow.
For example, an unrealistically inflated template size mandates according parameters for all functions which deal with the data in its affine-determined geometric space.
Additionally, if the template axis orientation deviates by more than \SI{45}{\degree} from the image to be registered (or, as is more often the case, an axis is flipped), the global maximum of the first (rigid) registration steps may not be correctly determined, and the image would then be skewed and nonlinearly deformed to match  the template at an incorrect orientation.
As such, template quality needs to be ascertained, and a workflow-compliant default should be provided.

Our recommended template (see \cref{fig:dsu}) is derived from the DSURQE template --- of the Toronto Hospital for Sick Children Mouse Imaging Center \cite{dsu}.
The geometric origin of this derived template corresponds to the Bregma landmark, and provides integration with histological atlases and surgical procedures (commonly using Bregma for reference).
The template is in the canonical orientation of the NIfTI format, RAS (left$\rightarrow$Right, posterior$\rightarrow$Anterior, inferior$\rightarrow$Superior), and has a coronal slice positioning reflective of both the typical animal head position in MR scanners and in stereotactic surgery frames.
The template is provided in \SI{40}{\micro\meter} and \SI{200}{\micro\meter} isotropic resolutions; 
and all of its associated mask and label files are identified with the prefix \textcolor{mg}{\texttt{dsurqec}} in the template packages.

We bundle the aforementioned MR template with two additional histological templates, derived from the Australian Mouse Brain Mapping Consortium (AMBMC) \cite{amb}, and the Allen Brain Institute (ABI) \cite{abi} templates.
While these atlases suffer from shortcomings listed under the \nameref{sec:bg}~section, we include the AMBMC template due to its extra long rostrocaudal coverage, and the ABI atlas due to its role as the reference atlas for numerous gene expression and projection maps.
We reorient the AMBMC template from its original RPS orientation to the canonical RAS, and apply an RAS orientation to the orientation-less ABI template after converting it to NIfTI from its original NRRD format.
These atlases are also made available in \SI{40}{\micro\meter} and \SI{200}{\micro\meter} isotropic resolutions, and the corresponding files are prefixed with \textcolor{mg}{\texttt{ambmc}} and \textcolor{mg}{\texttt{abi}}, respectively.

Additionally, we provide templates in the historically prevalent but incorrect, RPS orientation, and with the historically prevalent tenfold increase in voxel size.
These templates are derived from the DSURQE and AMBMC templates, and are prefixed with \textcolor{mg}{\texttt{ldsurque}} and \textcolor{mg}{\texttt{lambmc}}, respectively.

Lastly, due to data size considerations, we distribute \SI{15}{\micro\meter} isotropic versions of all atlases available at this resolution (AMBMC and its legacy derivative, as well as ABI) in a separate package.
The two packages which we thus distribute are called \textcolor{mg}{\texttt{mouse-brain-atlases}} and \textcolor{mg}{\texttt{mouse-brain-atlasesHD}};
and up-to-date versions of these archives can be reproduced via a FOSS script collection (which handles download, reorienting, and resampling, and was written and released for the purpose of this publication).

For the comparisons performed in this text, the \textcolor{mg}{\texttt{dsurqec}} and \textcolor{mg}{\texttt{ldsurqec}} template variations (which contain the same data matrix, but are matched to the orientation and size requirements of the \cref{fig:wfgg} and \cref{fig:wfgl} workflows, respectively) are referred to as the “Generic” template.
Analogously, the \textcolor{mg}{\texttt{ambmc}} and \textcolor{mg}{\texttt{lambmc}} template variations are referred to as the “Legacy” template.
%These naming choices derive from the to the introduction as the default template for the Generic SAMRI preprocessing workflow of the former and the prevalent usage in the Legacy SAMRI preprocessing workflow of the latter template.

\subsection{Testing Dataset}

\py{pytex_tab('scripts/stim_table.py',
		label='stim',
		caption='Stimulation protocol, as delivered during functional scans.
			Stimulus event spacing and parameters are constant across scans, but exact onset time is variable in the \SI{10}{\second} magnitude range due to scanner adjustment time variability.',
		options_pre='\\scriptsize \\centering \\resizebox{\\columnwidth}{!}{',
		data='data/JogB.tsv',
		options_post='}',
		)}


For the quality control of the workflows, a dataset with an effective size of 102 scans is used.
Data from 11 adult animals is included, each animal being recorded on up to 5 sessions (repeated at 14 day intervals).
Each session contains an anatomical scan and two functional scans --- with Blood-Oxygen Level Dependent (BOLD) and Cerebral Blood Volume (CBV) contrast, respectively (for a total of 68 functional scans).

Anatomical scans are acquired via a TurboRARE sequence, with a RARE factor of 8, an echo time (TE) of \SI{7}{\milli\second} and a repetition time (TR) of \SI{2500}{\milli\second}, sampled at a sagittal resolution of $\mathrm{\Delta x(\nu)=\SI{166.7}{\micro\meter}}$, a horizontal resolution of $\mathrm{\Delta y(\phi)=\SI{75}{\micro\meter}}$, and a coronal resolution of $\mathrm{\Delta z(t)=\SI{650}{\micro\meter}}$ (slice thickness of \SI{500}{\micro\meter}).
The functional BOLD and CBV scans are acquired with a flip angle of \SI{60}{\degree} and with $\mathrm{TR/TE = \SI{1000}{\milli\second}/\SI{15}{\milli\second}}$ and $\mathrm{TR/TE = \SI{1000}{\milli\second}/\SI{5.5}{\milli\second}}$, respectively.
Functional scans are sampled at $\mathrm{\Delta x(\nu)=\SI{312.5}{\micro\meter}}$, $\mathrm{\Delta y(\phi)=\SI{281.25}{\micro\meter}}$, and $\mathrm{\Delta z(t)=\SI{650}{\micro\meter}}$ (slice thickness of \SI{500}{\micro\meter}) --- the latter, in conjunction with an axial slice number of 14, amounting to an effective axial coverage of \SI{9.1}{\milli\meter}.
All aforementioned scans are acquired with a Bruker PharmaScan system (\SI{7}{\tesla}, \SI{16}{\centi\meter} bore), and an in-house T/R coil.

The measured animals are fitted with an optic fiber implant ($\mathrm{l=\SI{3.2}{\milli\meter} \ d=\SI{400}{\micro\meter}}$) targeting the Dorsal Raphe (DR) nucleus in the brain stem.
The nucleus is rendered sensitive to optical stimulation (see \cref{tab:stim} for protocol) by transgenic expression of Cre recombinase under the ePet promoter \cite{Scott2005} and viral injection of rAAVs delivering a plasmid with Cre-conditional expression of Channelrhodopsin and YFP ---
pAAV-EF1a-double floxed-hChR2(H134R)-EYFP-WPRE-HGHpA, a gift from Karl Deisseroth (Addgene plasmid \#20298).

The DR is stimulated via an Omicron LuxX 488-60 laser (\SI{488}{\nano\meter}) tuned to \SI{30}{\milli\watt} at contact with the fiber implant, according to the protocol listed in \cref{tab:stim}.

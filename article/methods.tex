\section{Methods}

For the quality control of the workflows, a dataset with an effective size of 102 scans is used.
Data from 11 adult animals
(\py{pytex_printonly('scripts/sex.py')},
\py{pytex_printonly('scripts/age.py')}
old at study onset)
is included, with each animal scanned on up to 5 sessions (repeated at 14 day intervals).
Each session contains an anatomical scan and two functional scans --- with Blood-Oxygen Level Dependent (BOLD) \cite{Ogawa1990} and Cerebral Blood Volume (CBV) \cite{Marota1999} contrast, respectively (for a total of 68 functional scans).

Anatomical scans were acquired via a TurboRARE sequence, with a RARE factor of 8, an echo time (TE) of \SI{21}{\milli\second}, an inter-echo spacing of \SI{7}{\milli\second}, and a repetition time (TR) of \SI{2500}{\milli\second}, sampled at a sagittal resolution of $\mathrm{\Delta x(\nu)=\SI{166.7}{\micro\meter}}$, a horizontal resolution of $\mathrm{\Delta y(\phi)=\SI{75}{\micro\meter}}$, and a coronal resolution of $\mathrm{\Delta z(t)=\SI{650}{\micro\meter}}$ (slice thickness of \SI{500}{\micro\meter}).
The field of view covered \SI[product-units=power]{20 x 9}{\mm} and was sampled via a $\mathrm{120 \times 120}$ matrix.
A total of 14 slices were acquired.

The functional BOLD and CBV scans were acquired using a gradient-echo Echo Planar Imaging (ge-EPI) sequence with a flip angle of \SI{60}{\degree} and with $\mathrm{TR/TE = \SI{1000}{\milli\second}/\SI{15}{\milli\second}}$ and $\mathrm{TR/TE = \SI{1000}{\milli\second}/\SI{5.5}{\milli\second}}$, respectively.
Functional scans were sampled at $\mathrm{\Delta x(\nu)=\SI{312.5}{\micro\meter}}$, $\mathrm{\Delta y(\phi)=\SI{281.25}{\micro\meter}}$, and $\mathrm{\Delta z(t)=\SI{650}{\micro\meter}}$ (slice thickness of \SI{500}{\micro\meter}).
The field of view covered \SI[product-units=power]{20 x 9}{\mm} and was sampled via a $\mathrm{64 \times 32}$ matrix.
A total of 14 slices were acquired.

All aforementioned scans were acquired with a Bruker PharmaScan system (\SI{7}{\tesla}, \SI{16}{\centi\meter} bore), and an in-house T/R coil.

The measured animals were fitted with an optic fiber implant ($\mathrm{l=\SI{3.2}{\milli\meter} \ d=\SI{400}{\micro\meter}}$) targeting the Dorsal Raphe (DR) nucleus in the brain stem.
The nucleus was rendered sensitive to optical stimulation by transgenic expression of Cre recombinase under the ePet promoter \cite{Scott2005} and viral injection of rAAVs delivering a plasmid with Cre-conditional expression of Channelrhodopsin and YFP ---
pAAV-EF1a-double floxed-hChR2(H134R)-EYFP-WPRE-HGHpA, a gift from Karl Deisseroth (\href{https://www.addgene.org/20298/}{Addgene plasmid \#20298}).

The DR was stimulated via an Omicron LuxX 488-60 laser (\SI{488}{\nano\meter}) tuned to \SI{30}{\milli\watt} at contact with the fiber implant, according to the protocol listed in \cref{tab:stim}.
The operation and stimulation procedure, as well as general picture of obtained activation is consistent with previous results \cite{Grandjean2019}, and is not further commented in this study.

All experimental procedures were approved by the Veterinary Office of the Canton of Zurich and done in accordance with the relevant regulations.

\subsection{Metrics}

For the current VCF implementation we define brain volume as estimated by the 66\textsuperscript{th} voxel intensity percentile of the raw scan before any processing.
The arbitrary unit equivalent of this percentile threshold is recorded for each scan and applied to all preprocessing workflow results for that particular scan, to obtain VCF estimates
 --- \cref{eq:vcf}, where $v$ is the voxel volume in the original space, $v^\prime$ the voxel volume in the transformed space, $n$ the number of voxels in the original space, $m$ the number of voxels in the transformed space, $s$ a voxel value sampled from the vector $S$ containing all values in the original data, and $s^\prime$ a voxel value sampled from the transformed data.

\begin{equation} \label{eq:vcf}
        V\!C\!F
        = \frac{v^\prime\sum_{i=1}^m [s^\prime_i \geq P_{66}(S)]}{v\sum_{i=1}^n [s_i \geq P_{66}(S)]}
        = \frac{v^\prime\sum_{i=1}^m [s^\prime_i \geq P_{66}(S)]}{v \lceil0.66n\rceil}
\end{equation}

The SCF metric is based on the ratio of smoothness before and after processing.
The smoothness measure is the full-width at half-maximum (FWHM) of the signal amplitude spatial autocorrelation function (ACF \cite{eklund2016cluster}).
Since fMRI data usually do not have a Gaussian-shaped spatial ACF, we use AFNI \cite{cox1996afni} to fit the following function in order to compute the FWHM ---
\cref{eq:acf}, where $r$ is the distance of two amplitude distribution samples, $a$ is the relative weight of the Gaussian term in the model, $b$ is the width of the Gaussian and $c$ the decay of the mono-exponential term \cite{cox2017fmri}.

\begin{equation} \label{eq:acf}
        ACF(r)
        = a * e^{ -r^{2}/ (2 * b^{2}) } + (1 - a) + e^{-r/c}
\end{equation}

We examine statistical power as relevant for the MS metric via the negative logarithm of first-level p-value maps.
This produces voxelwise statistical estimates for the probability that a time course could --- by chance alone --- be at least as well correlated with the stimulation regressor as the voxel time course measured.
We compute the per-scan average of these values as seen in \cref{eq:ms}, where $n$ represents the number of statistical estimates in the scan, and $p$ is a p-value.

\begin{equation} \label{eq:ms}
        M\!S = \frac{\sum_{i=1}^n -log(p_i)}{n}
\end{equation}

\subsection{Software}

The workflow functions make use of the Nipype \cite{nipype} package, which provides high-level workflow management and execution features.
Via this package, functions provided by any other package can be encapsulated in a node (complete with error reporting and isolated re-execution support) and integrated into a directed workflow graph.
Parallelization can also be managed via a number of execution plugins, allowing scalability.
Most importantly, Nipype can generate graph descriptor language (DOT) summaries, as well as visual workflow representations suitable for operator inspection, graph theoretical analysis, and programmatic comparison between workflow variants.

Via Nipype, we utilize basic MRI preprocessing functions from the FSL package \cite{fsl} and registration functions from the ANTs package \cite{ants}.
While there is theoretically no limit to the number of external packages usable with Nipype, we constrain our choice as much as possible in order to minimize the dependency graph.
The choice of the ANTs package (in addition to FSL, which also provides registration functions) owes to the package's functions being more highly parameterized.
This feature allows us to avoid maladaptive optimization choices, and instead fine-tune the registration to the overarching characteristics of the brain type at hand.
Additionally, we have implemented a number of functions in our workflow directly, e.g. to read BIDS \cite{bids} inputs, and perform dummy scans management.

For Quality Control we distribute as part of this publication additional workflows using the NumPy \cite{numpy}, SciPy \cite{scipy}, pandas \cite{pandas}, and matplotlib packages \cite{matplotlib}, as well as Seaborn \cite{seaborn} for plotting, and Statsmodels \cite{statsmodels} for top-level statistics, using the HC3 heteroscedasticity consistent covariance matrix \cite{long2000}.
Specifically, distribution densities for plots are drawn using the Scott bandwidth density estimator \cite{Scott1979}.

\section{Data and Code Availability}

The data archive relevant for this article is freely available \cite{mlebe_bidsdata}, and automatically accessible via the Gentoo Linux package manager.
The code relevant for reproducing this document is also freely available \cite{source}, as are its dependencies, and most prominently, SAMRI \cite{samri}, the package via which the herein described workflows are distributed.

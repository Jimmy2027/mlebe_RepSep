\section{Methods}\label{sec:methods}
For the benchmarking of the two workflows, the same methods that are described in the original paper have been applied in this work.
A more detailed description can be found there.

\begin{sansmath}
\py{pytex_subfigs(
        [
                {'script':'scripts/classifier/uprepex.py', 'label':'exunprepro', 'conf':'article/1col.conf', 'options_pre':'{.48\\textwidth} \\vspace{-2em}',
                        'options_pre_caption':'\\vspace{0.1em}',
                        'options_post':'\\vspace{1em}',
                        'caption':'Example of an unpreprocessed slice.'
                        ,},
                {'script':'scripts/classifier/prepex.py', 'label':'exprepro', 'conf':'article/1col.conf', 'options_pre':'{.48\\textwidth} \\vspace{-2em}',
                        'options_pre_caption':'\\vspace{0.1em}',
                        'options_post':'\\vspace{1em}',
                        'caption':'Example of a preprocessed slice.'
                        ,},
                ],
        caption='\\textbf{The preprocessing removes the mask there, where the image-pixelvalues are 0.}\\
        Plots of the same image, superposed with the template mask, with and without preprocessing.
        ',
        label='fig:prepro_examples',
        )}
\end{sansmath}

\begin{sansmath}
\py{pytex_fig('scripts/classifier/plt_trainset.py',
        conf='article/2*8plot.conf',
        label='trainset',
        caption='
        Augmented samples from the Training set.
        ',
        multicol=True,
        )}
\end{sansmath}

\begin{sansmath}
\py{pytex_fig('scripts/classifier/show_blacklist.py',
        conf='article/2*8plot.conf',
        label='blacklist_example',
        caption='
                \\textbf{Slices where the mask includes too much outer-brain intensities are excluded from the data set.}\\
                Examples from the slices that were excluded from the data set. The mask is shown in blue, on top of the brain image.
                ',
        multicol=True,
        )}
\end{sansmath}

\subsection{Model}
As the architecture of the classifier, the U-Net from Ronneberger et al \cite{ronneberger_u-net:_2015} was chosen based on its high performance in the field of biomedical image segmentation.
This is a convolutional neural network that consists of a contracting path that captures context in addition to a symmetric expanding path that enables precise localization.
Localization in this context means that a class label is assigned to each pixel.
We used the attention gated U-Net implementation from Ozan Oktay et al. \cite{oktay_attention} for which the code is publicly available \cite{oktay_ozan-oktayattention-gated-networks_2020}.
Additionally to the original U-Net structure, their implementation has attention gates in the expanding part which weight the information coming from the symmetric counterpart.
The additional parameters in these attention gates allow the model to learn which region of the image is important for specific tasks and to suppress irrelevant regions.
In our use case this implementation helped reduce false positive classifications of high intensities, outside of the mouse brain region.

The model was trained using the Dice loss, which is computed from the Dice score.
It calculates the similarity of two binary samples X and Y with
\begin{equation}\label{eqDcoef}
D_{coef} = \frac {2|X\cap Y| + \epsilon}{|X|+|Y| + \epsilon}
\end{equation}
where a smoothing factor $\epsilon$ of 0.01 is used.

It is a quantity ranging from 0 to 1 that is to be maximized.
The loss is then calculated with $1-D_{coef}$.

Because many more pixels in the masks are 0 than 1, there is a class imbalance problem.
This is a problem because in this case a false positive gives a much higher loss than a false negative.
For example, predicting only black would give an acceptable loss, while predicting only white pixels would not.
Using the Dice coefficient as a loss function for training should make it invariant to this class imbalance problem as stated by Fausto Milletari et al. in \cite{milletari_v-net:_2016}.

\subsection{Data Set} \label{subsec:Data Set}
The data set consists of 3D MR images taken from an aggregation of three studies: irsabi \cite{irsabi_bidsdata}, opfvta \cite{ioanas_whole-brain_nodate}, drlfom \cite{ioanas_effects_nodate} and other unpublished data, acquired with similar parameters.

The irsabi data set consists of 102 scans coming from 11 adult animals, each scanned in up to 5 sessions with a 7T Bruker PharmaScan.
The sessions were repeated at 14 days intervals, each containing one anatomical (echo-time: 21ms, inter-echo spacing: 7ms, repetition time (TR): 2500ms) and two functional (CBV and BOLD with a flip angle of 60°) scans.
The functional scans were sampled at $\mathrm{\Delta x(\nu)=\SI{312.5}{\micro\meter}}$, $\mathrm{\Delta y(\phi)=\SI{281.25}{\micro\meter}}$, and $\mathrm{\Delta z(t)=\SI{650}{\micro\meter}}$ (slice thickness of \SI{500}{\micro\meter}).

The opfvta data set consists of \py{boilerplate.get_nmbrScans_from_dataselection('opfvta')} scans coming from \py{boilerplate.get_nmbrSubject_from_dataselection('opfvta')} adult animals, each scanned in up to \py{boilerplate.get_max_numbrSession_from_dataselection('opfvta')} sessions with a 7T Bruker PharmaScan.
The sessions were repeated at ??? days intervals, each containing one anatomical (echo-time: 30ms, inter-echo spacing: 10ms, repetition time (TR): 2950ms) and a functional (CBV with a flip angle of 60°) scan.
The functional scans were sampled at $\mathrm{\Delta x(\nu)=y(\phi)=\SI{75}{\micro\meter}}$ and a slice thickness of $\mathrm{\Delta z(t)=\SI{450}{\micro\meter}}$.

The drlfom data set consists of \py{boilerplate.get_nmbrScans_from_dataselection('drlfom')} scans coming from \py{boilerplate.get_nmbrSubject_from_dataselection('drlfom')} adult animals, each scanned in up to \py{boilerplate.get_max_numbrSession_from_dataselection('drlfom')} sessions with a 7T Bruker PharmaScan.
The sessions were repeated at ??? days intervals, each containing one anatomical (echo-time: 30ms, inter-echo spacing: 10ms, repetition time (TR): 2950ms) and a functional (CBV with a flip angle of 60°) scan.
The functional scans were sampled at $\mathrm{\Delta x(\nu)=y(\phi)=\SI{225}{\micro\meter}}$, and a slice thickness$\mathrm{\Delta z(t)=\SI{450}{\micro\meter}}$.

The measured animals were fitted with an optic fiber implant ($\mathrm{l=\SI{3.2}{\milli\meter} \ d=\SI{400}{\micro\meter}}$) targeting the Dorsal Raphe (DR) nucleus in the brain stem.
Using this dataset shows that the classifier is robust to these types of experiment setups.
%todo is this true for all the datasets?
Images from the irsabi study are only used for quality control of the registration and are thus unknown to the classifier.
It is the same dataset that was used to benchmark the Generic workflow in the original paper and thus allows for a better estimation of the general performance of our improved pipeline.

The images are transformed into a standard space using a template mask via SAMRI \cite{noauthor_ibt-fmi/samri_2019} and are thus defined in the same affine space.
SAMRI is a data analysis package of the ETH/UZH Institute for Biomedical Engineering.
It is equipped with an optimized registration workflow and standard geometric space for small animal brain imaging \cite{ioanas_optimized_2019}.

Because of variance in mouse brain anatomy and in the experiment setup, some of the transformed data do not overlap perfectly with the reference template.
To filter these images out, most of the incongruent volumes were removed manually from the data set.

For the registration of the images, a padding was needed to make the originally not affine space affine.
As a result, the 3D volumes present many zero-valued slices, some of them overlapping with the mask.

Since it is not wanted for the model to predict a mask on black slices, the mask is set to zero where the image is zero-valued.
This has also the advantage of bringing variance into the template.
Because some pixels representing the brain tissue are zero-valued, holes result from this operation.
To patch these, the function \textcolor{mg}{\texttt{$binary\_fill\_holes$}} from scipy.ndimage.morphology \cite{noauthor_multi-dimensional_nodate} is used.
An example of the preprocessing can be seen in \cref{fig:prepro_examples}.

In the coronal view, each slice of the transformed data is originally of shape (63, 48), matching the reference space resolution of \SI{200}{\micro\metre}.
It is then reshaped into \py{boilerplate.get_training_shape('tuple')} by first zero-padding the smaller dimension to the same size as the bigger one and then reshaping the image into \py{boilerplate.get_training_shape()} using the function \textcolor{mg}{\texttt{$cv2.resize$}} from the opencv python package \cite{noauthor_opencv-python_nodate}.

Finally, the images are normalized by first clipping them from the minimum to the \nth{99} percentile of the data to remove outliers and then divided by the maximum.

The data set is separated into Training, Validation and Test sets such that 90\% of the total data are used for training and validation while 10\% are used for testing.
This is done with the help of the function \textcolor{mg}{\texttt{$train\_test\_split$}} from the package sklearn.model$\_$selection \cite{scikit-learn}.
The Validation set is used for the optimization of hyperparameters while the Test set is used as a measure of extrapolation capability.
The irsabi data was additionally added to the test set.

\subsection{Data Augmentation} \label{Data Augmentation}

Because of diverse settings in the experiment setup, including animal manipulations causing artifacts, MR image quality can differ substantially between labs and even individual study populations.
To account for these variations, we apply an extensive set of transformations to our data.
This includes rotations of up to 20$^{\circ}$, a zoom range of -0.2 to +0.1, a random bias field added on the images and horizontal as well as vertical flips.
Additionally a gaussian noise is added to the images.

This not only increases the data set size but also makes it more representative of the general data distribution of mice brain MR images and results in a model with a better generalization capability.

\subsection{Training}
The model was trained on 3D volumes in the coronal view.
It was chosen over the axial one, because the shapes of the masks are much simpler in the coronal view and thus easier to learn for the network.

Additionally, the coronal view has the advantage of higher resolution as the MR images were recorded coronally.

\subsection{Masking}
To improve the SAMRI registration workflow, an additional node is implemented where the images are masked, such that only the brain region remains.
The image is first resampled into the resolution of the template space, which has a voxel size of $0.2\times 0.2 \times 0.2$.
This is done with the \textcolor{mg}{\texttt{$Resample$}} command from the FSL library which is an analysis tool for FMRI, MRI and DTI brain imaging data \cite{fsl}.
Then, the image is preprocessed using the operations described in \cref{subsec:Data Set}.
Since the classifier was trained to predict on images of shape \py{boilerplate.get_training_shape('tuple')}, the input needs to be reshaped.
The predictions of the model are reconstructed to a 3D mask via the command \textit{Nifit1Image} from the neuroimaging python package nibabel \cite{noauthor_neuroimaging_nodate}.
This is done using the same affine space as the input image.
The latter is then reshaped into the original shape inverting the preprocessing step, either with the opencv resize method or by cropping.
Additionally, the binary mask is resampled into its original affine space, before being multiplied with the brain image to extract the ROI.

\subsection{Metrics}

The VCF uses the 66\textsuperscript{th} voxel intensity percentile of the raw scan before any processing as definition of the brain volume.
The VCF is then obtained with \cref{eq:vcf}, where $v$ is the voxel volume in the original space, $v^\prime$ the voxel volume in the transformed space, $n$ the number of voxels in the original space, $m$ the number of voxels in the transformed space, $s$ a voxel value sampled from the vector $S$ containing all values in the original data, and $s^\prime$ a voxel value sampled from the transformed data.

\begin{equation} \label{eq:vcf}
        V\!C\!F
        = \frac{v^\prime\sum_{i=1}^m [s^\prime_i \geq P_{66}(S)]}{v\sum_{i=1}^n [s_i \geq P_{66}(S)]}
        = \frac{v^\prime\sum_{i=1}^m [s^\prime_i \geq P_{66}(S)]}{v \lceil0.66n\rceil}
\end{equation}

The SCF metric is based on the ratio of smoothness before and after processing.
It is obtained via \cref{eq:acf}, where $r$ is the distance of two amplitude distribution samples, $a$ is the relative weight of the Gaussian term in the model, $b$ is the width of the Gaussian and $c$ the decay of the mono-exponential term \cite{cox2017fmri}.

\begin{equation} \label{eq:acf}
        ACF(r)
        = a * e^{ -r^{2}/ (2 * b^{2}) } + (1 - a) + e^{-r/c}
\end{equation}

The for the MS relevant statistical power is obtained via the negative logarithm of first-level p-value maps.
Voxelwise statistical estimates for the probability that a time course could --- by chance alone --- be at least as well correlated with the stimulation regressor as the voxel time course measured are averaged via \cref{eq:ms}, where $n$ represents the number of statistical estimates in the scan, and $p$ is a p-value.

\begin{equation} \label{eq:ms}
        M\!S = \frac{\sum_{i=1}^n -log(p_i)}{n}
\end{equation}

\subsection{Statistics}

In the results section, all statistics are presented with respect to the distributions of the absolute distances to 1, i.e. |1 - Metric|.
Based on a Likelihood Ratio Test, we chose models that do not examine the Workflow- Contrast interaction.
The full summaries of the analysis can be seen in tables \cref{table1}, \cref{table2}, \cref{table3} and \cref{table4}.
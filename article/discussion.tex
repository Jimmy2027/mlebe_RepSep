\section{Discussion}
\subsection{Workflow and Template Recommendation}

We conclude from the evaluation that the workflow and template design presented herein offer significant advantages in terms of reducing coverage overestimation and region misassignment.
This is most clearly evident from the Volume Conservation results (\cref{fig:vc}) where the joint usage of the SAMRI Generic workflow and Generic template decisively outperform all other combinations of the multivariate analysis.
The spatial robustness thus evident is also revealed in a qualitative examination of higher-level functional maps (as seen in \cref{fig:m}, where the combination of the Generic workflow using the Generic template provides, unlike all others, accurate coverage across both contrasts).
We note that these benefits are provided entirely without compromising statistical power (\cref{fig:ms}), and that this holds for both CBV and BOLD contrasts (\cref{fig:vccv,fig:mscv}).
Additionally, we note that the Generic processing workflow is more consistent, as shown in notable reductions of the standard deviation for both VCS and MS (\cref{fig:vcv,fig:msv}).

Further supporting this recommendation is that these features are augmented by design benefits such as added transparency and parameterization of the workflow (which can more easily be inspected and further improved or customized by the end user), veracity of resulting data headers, and resulting coordinates more meaningful for surgery and histology.

\subsection{Quality Control}

We note that the Volume Change Factor (VCF) is a remarkably simple, powerful, and robust QC tool for registration performance.
It provides a good quantitative estimate of distortion features which are salient in qualitative operator inspection (e.g. in \cref{fig:m}).
Indeed, we believe that this metric is in itself interesting and could itself be further optimized (e.g. by developing percentile selection heuristics based on a priori documented data distortions).

Conversely, we note that global statistical power is not --- at least in the range of workflows at hand --- sensitive to registration.
It is thus not a reliable metric for optimization, though sadly, it may be the most prevalently used if results are only inspected at a higher level.
In isolated cases, even, as shown by the positive interaction effect of the Legacy workflow level and the CBV contrast for the comparison in \cref{fig:mscv}, usage of this metric alone may give a misleading indication.
We would not discount this measure entirely, however, as it is strongly sensitive to workflow variations which we have consciously not included in this comparison (such as registration interpolation method).

Overall we suggest that a VCF comparison, coupled with visual inspection of a small number of omnibus statistical maps is a feasible and sufficient tool for benchmarking workflows, with MS usable as an additional sanity check.
Further, we recommend the data herein presented for workflow benchmarking, as it includes (a) multiple sources of variation (contrast, session, subjects), (b) functional activity with broad coverage but spatially distinct features, (c) significant distortions due to implant presence --- which are appropriate for testing the workflow robustness.
Not least of all, we hope that the RepSeP-compilant executable source code, which reliably reproduces this document, will make it easier not only to fully inspect and understand our results, but also to apply all of our tests to further data or further workflows.

\subsection{Limitations}

We acknowledge that our recommendation can only extend as far to favour of the Generic template and the Generic SAMRI workflow as an indivisible unit.
More fine-grained processing steps cannot be reliably advocated or dismissed based solely on the holistic evaluation presented herein.
Further, we hold that the two workflows discussed represent discrete processing principles and are not arbitrarily mixable (e.g. intensity manipulations go hand in hand with the masking choice, and the structural intermediary goes hand in hand with the registration interface choice).
Thus, in support of \textit{individual} nodes in the processing workflows (\cref{fig:wfg}), we only make theoretical claims.

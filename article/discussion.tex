While we acknowledge that individual processing steps cannot be reliably recommended or dismissed based solely on our holistic workflow comparison, we hold that the two workflows discussed represent discrete processing principles which are not arbitrarily mixable (e.g. intensity manipulations go hand in hand with the masking choice, and the structural intermediary goes hand in hand with the registration interface choice).
Therefore, we insist that our results represent a general recommendation for all processing steps of our new “SAMRI Generic” workflow and against all differing processing steps seen in the “SAMRI Legacy” or analogous workflows.

Our registration via a structural intermediary allows the explicit and parameterized control of the susceptibility artefact data reconstruction method - as anatomical variation and susceptibility artefacts do not become convoluted.
We make the choice of performing no reconstruction, as this preserves the highest data veracity (at the cost of effective comparable voxels across animals).

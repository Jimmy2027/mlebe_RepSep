\section{Discussion}

The classifier improves the volume conservation, smoothness conservation, and session-to-session consistency of the SAMRI Generic workflow in terms of accuracy and precision.

Region assignment validity is also revealed in a qualitative examination of higher-level functional maps (\cref{fig:m}), where both the Generic and the Generic* workflow provide accurate coverage of the sampled volume for both BOLD and CBV fMRI data.

These benefits of the classifier are robust to the functional contrast (\cref{fig:vccv,fig:sccv,fig:mscv}), with the Generic* workflow being less or equally susceptible to the contrast variable, when compared to the Generic workflow.

The classifier improves the workflow, while maintaining transparency and parameterization as well as the veracity of resulting data headers.
The complete workflow of this report is fully reproducible and thus easily falsifiable.
We make public the functions used for the masking in the workflow as well as those used to train the classifier, through the mlebe python package.

Our workflow has the advantage that the performance of a Neural Network can increase when trained further with new data.
A user could easily recreate the steps described herein.
Registering new data with the Generic workflow can increase the size of the training data set of the classifier.
After removing bad registrations, the latter can be trained again, which will improve its generalisation capability.
It is worth noting that classifiers can be shared between users to maximize efficiency.
Another advantage of the trainability of the classifier and the openly published code is that this workflow can be adapted to a wast variety of data types.

\subsection{Quality Control}

We recommend reusing the presented data for workflow benchmarking, as they include (a) multiple sources of variation (contrast, session, subjects), (b) functional activity with broad coverage but spatially distinct features, and (c) significant distortions due to implant properties --- which are appropriate for testing workflow robustness.
In addition to the workflow code \cite{samri}, we openly release the re-executable source code \cite{source} for all statistics and figures in this document.
It is thus not just the novel method, but also the benchmarking process which is fully transparent and reusable for further data.

\subsection{Conclusion}

We present a remodeled version of the SAMRI Genetic registration workflow, which offers several advantages.
In depth multivariate comparison with the original revealed superior performance of the SAMRI Generic* workflow in terms of volume and smoothness conservation, as well as variance structure across subjects and sessions.
The easily accessible, optimized registration parameters of the SAMRI Generic Workflow as well as the open source code to the classifier training functions make the pipeline transferable to any other imaging applications.
The open source software choices in both the workflow and this article's source code empower users to better verify, understand, remix, and reuse our work.

\section{Background}
Functional magnetic resonance imaging (fMRI) gives an indirect measurement of brain activity by being sensitive to the change of blood flow.
It is one of the most prominent neuroimaging tool for many applications, such as drug discovery \cite{borsook_role_2006} and neuromodeling \cite{friston_dynamic_2003}.
For fMRI studies, it is necessary that all scans lie in a standard reference frame in order to make meaningful comparisons across the subjects.
The common coordinate system enables a statistical evaluation of the likelihood of consistent activation across a group or, in other contexts, the differences in anatomy between two groups.
Because of variability both in animal anatomy and in animal preparation, the original MR acquired images are not defined in a common template space.
To solve this issue, scans need to be remapped to a reference frame via registration \cite{maintz_overview_nodate, sotiras_deformable_2013}.
As reported by Ioanas et al. \cite{ioanas_optimized_2019}, the legacy approach for mouse-brain image registration is to modify the data in order to conform to pre-existing functions, designed and optimized for human brain imaging.
This requires the mouse-data to be adapted to the processing function instead of vice-versa.
\cite{ioanas_optimized_2019} establishes a novel workflow defined as generic, specifically designed for mouse brain imaging, and benchmarks it against the legacy procedure.
While the reported performance increase is considerable, registration is nonetheless influenced by intensity variations outside the brain region.
In-vivo as well as ex-vivo MRI head scans, present higher variability in the viscerocranial and extracranial tissue than in the neurocranium and the brain region of interest.
Usage of unmasked (i.e. non brain extracted) data as done by the generic method, can thus lead to stretching or skewing of the brain during the registration process.
Computing the transformation solely on the brain volume removes disturbances induced by intensity variations outside the brain region and further improves registration quality.

In recent years it has been shown that convolutional neural networks give the best results for semantic image segmentation in terms of precision and flexibility \cite{geng_survey_2018}.
Especially the U-Net architecture from \cite{ronneberger_u-net:_2015} is to this day one of the most popular in the field of biomedical image segmentation.
Training a neural network into a classifier is a supervised method.
This means that the model needs to learn its parameters based on observations of labeled data.
Manually creating annotations as required to train a deep-learning classifier for high-resolution data is often infeasible, since it requires manual expert segmentation of vast amounts of slices.
In the medical domain especially, human labeled data is expensive to acquire and thus very scarce.
A much more widely applicable approach is to train the network using the template mask as label together with registered scans.
Registration is not as precise as human labeling, but it is automatic and does not depend on expert input.
\cite{imperfect_datasets, imperferct_segmentaion_labels} show that deep learning methods can indeed show satisfiable results when trained with imperfect training data.
While our purpose was to create a workflow that generates better masks than the one from the template space, we show that the latter can be used as training data for the deep-learning model, by applying small changes to it.


In this study we investigate whether and in how far reliable classification can be obtained from imperfect training data and whether preclinical image masking improves an optimized registration workflow.
We provide the methods as a free and open source softare (FOSS) package \cite{mlebe} as well as the functions needed for the data analysis in this article as a RepSeP document \cite{repsep}.
We evaluate the effects of our classifier on a full-fledged registration workflow via the benchmarking algorithms from \cite{ioanas_optimized_2019}.
\section{The Classifier}

%As part of the analysis of high-field mouse MRI data, relevant brain tissue needs to be selected via a mask.
%For this process to be performed automatically, brain voxels need to be classified based both on their signal intensity and position in the image.
%Nowadays, deep neural networks are the state-of-the-art methods for tissue segmentation in biomedical imaging, and thus constitute a promising method for preclinical neuroscience.


\begin{sansmath}
\py{pytex_subfigs(
        [
                {'script':'scripts/classifier/prepex.py', 'label':'exprepro', 'conf':'article/1col.conf', 'options_pre':'{.48\\textwidth} \\vspace{-2em}',
                        'options_pre_caption':'\\vspace{-1.5em}',
                        'options_post':'\\vspace{1em}',
                        'caption':'Example of a preprocessed slice.'
                        ,},
                {'script':'scripts/classifier/uprepex.py', 'label':'exunprepro', 'conf':'article/1col.conf', 'options_pre':'{.48\\textwidth} \\vspace{-2em}',
                        'options_pre_caption':'\\vspace{-1.5em}',
                        'options_post':'\\vspace{1em}',
                        'caption':'Example of an unpreprocessed slice.'
                        ,}
                ],
        caption='\\textbf{The preprocessing removes the mask there, where the image-pixelvalues are 0.}\\
        Plots of the same image, superposed with the template mask, with and without preprocessing.
        ',
        label='fig:prepro_examples',
        )}
\end{sansmath}
%\begin{sansmath}
%\sessionpy{pytex_fig(
%        'scripts/prepex.py',
%        conf='article/1col.conf',
%        caption='
%                Distribution densities of t-statistics, showing the regions were VTA structural projections exceed functional activation most strongly.
%                \\vspace{-1em}
%                ',
%        label='dffn',
%        options_pre='\\centering\n',
%        environment='figure',
%        figure_format='pdf',
%        )}
%\end{sansmath}
\subsection{Model}
As the architecture of the classifier, the U-Net from Ronneberger et al \cite{ronneberger_u-net:_2015} was chosen based on its high performance in the field of biomedical image segmentation.
This is a convolutional neural network that consists of a contracting path that captures context in addition to a symmetric expanding path that enables precise localisation.
Localisation in this context means that a class label is assigned to each pixel.
We used the U-Net implementation from zhixuhao \cite{zhixuhao_zhixuhao/unet_2020}, written in Keras.
Keras is a high-level neural networks API, written in Python and capable of running on top of TensorFlow, CNTK, or Theano.
It allows for a easily readable code and makes it thus easier to reproduce.

The implementation of the U-Net from zhixuhao has two drop-out layers in addition to the original one.
A drop-out layer randomly sets a fraction of input units from the layer to 0 at each update during training time.
The set fraction rate is 0.5.
It is known that dropout helps prevent overfitting and greatly improves the performance of deep learning models \cite{srivastava_dropout:_nodate}.

Three losses were tested for the training of the model, namely the Dice-loss, the binary-cross-entropy loss and the sum of both.

The Dice-loss is computed from the Dice score. It calculates the similarity of two binary samples X and Y with
\begin{equation}\label{eqDcoef}
D_{coef} = \frac{2|X\cap Y|}{|X|+|Y|}   
\end{equation}

It is a quantity ranging from 0 to 1 that is to be maximised.
The loss is then calculated with $1-D_{coef}$.
Because the Dice loss is not differentiable, small changes need to be made.
In our case, the two samples to be compared are normalised, grey valued images  and are thus not binary but have values between 0 and 1. 
Additionally, instead of using the logical operation \textit{and}, element wise products are used to approximate the non-differentiable intersection operation.
To avoid a division by zero, $+1$ is added on the numerator and denominator.

Because many more pixels in the masks are 0 than 1, there is a class imbalance problem.
It is a problem, because in this case a false positive gives a much higher loss than a false negative.
For example, predicting only black would give an acceptable loss, while predicting only white pixels would not.
Using the Dice coefficient as a loss function for training should make it invariant to this class imbalance problem as stated by Fausto Milletari et al. in \cite{milletari_v-net:_2016}.

The binary cross-entropy loss, also called Log loss, is defined by:
\begin{equation}\label{eq_bincross}
    H_p (q) = -\frac{1}{N} \sum ^N _{i=1} y_i \cdot log(p(y_i))+(1-y_i) \cdot log(1-p(y_i))
\end{equation}

For pixel values of 0 and 1, it adds $log(p(y))$ for each white pixel (y=1) and $log(1-p(y))$ for every black pixel (y=0) to the loss.

We quickly realised that the Dice-loss gives the best results for our problem and therefore used it to train the model. )TODO:compare results)

%\begin{sansmath}
%\sessionpy{pytex_figs(
%        'scripts/prepro_example.py',
%        conf='article/1col.conf',
%        caption='
%                Distribution densities of t-statistics, showing the regions were VTA structural projections exceed functional activation most strongly.
%                \\vspace{-1em}
%                ',
%        label='dffn',
%        options_pre='\\centering\n',
%        environment='figure',
%        figure_format='pdf',
%        )}
%\end{sansmath}
%\begin{sansmath}
%\py{pytex_subfigs(
%        [
%                {'script':'scripts/prepex.py', 'label':'vcv', 'conf':'article/1col.conf',                              'options_pre':'{.48\\textwidth} \\vspace{-2em}',
%                        'options_pre_caption':'\\vspace{-1.5em}',
%                        'options_post':'\\vspace{1em}',
%                        'caption':'Comparison across workflows and target templates, considering both BOLD and CBV functional contrasts.'
%                        ,},
%                {'script':'scripts/uprepex.py', 'label':'vccv','conf':'article/1col.conf',                                'options_pre':'{.48\\textwidth} \\vspace{-2em}',
%                        'options_pre_caption':'\\vspace{-1.5em}',
%                        'options_post':'\\vspace{1em}',
%                        'caption':'Comparison across workflows and functional contrasts, considering only matching template-workflow combinations.'
%                        ,},
%
%                ],
%        caption='\\textbf{The SAMRI Generic workflow and template optimally and reliably conserve volume and smoothness --- unlike the Legacy workflow and template.}
%        Plots of three target metrics, with coloured patch widths estimating distribution density, solid lines indicating the sample mean, and dashed lines indicate the inner quartiles.
%        ',
%        label='fig:vc',
%        )}
%\end{sansmath}

\subsection{Data Set} \label{subsec:Data Set}
The data set consists of 3D MR images taken from an aggregation of three studies; irsabi , opfvta \cite{ioanas_whole-brain_nodate}, drlfom \cite{ioanas_effects_nodate} and other unpublished data, acquired with similar parameters.
%\todo{cite}

The images are transformed into a standard space with one defined mask via SAMRI \cite{noauthor_ibt-fmi/samri_2019} and are thus defined in the same affine space.
SAMRI is a data analysis package of the ETH/UZH Institute for Biomedical Engineering.
It is equipped with an optimized registration workflow and standard geometric space for small animal brain imaging \cite{ioanas_optimized_2019}.

Because of variance in mouse brain anatomy and in the experiment setup, some of the transformed data do not overlap perfectly with the reference template.
To filter these images out, most of the incongruent slices were removed manually from the data set.

For the registration of the images, a padding was needed to make the originally not affine space affine. 
Due to this, the 3D volumes present many zero-valued slices, some of them overlapping with the mask.

%\begin{sansmath}
%\py{pytex_subfigs(
%        [
%                {'script':'scripts/vc_violin.py', 'label':'vcv', 'conf':'article/1col.conf', 'options_pre':'{.48\\textwidth} \\vspace{-2em}',
%                        'options_pre_caption':'\\vspace{-1.5em}',
%                        'options_post':'\\vspace{1em}',
%                        'caption':'Comparison across workflows and target templates, considering both BOLD and CBV functional contrasts.'
%                        ,}
%                ],
%        caption='\\textbf{The SAMRI Generic workflow and template optimally and reliably conserve volume and smoothness --- unlike the Legacy workflow and template.}
%        Plots of three target metrics, with coloured patch widths estimating distribution density, solid lines indicating the sample mean, and dashed lines indicate the inner quartiles.
%        ',
%        label='fig:vc',
%        )}
%\end{sansmath}
Since it is not wanted for the model to predict a mask on black slices, the mask is set to zero where the image is as well.
Because some pixels representing the brain tissue are zero-valued, holes result from this operation.
To patch these, the function \textcolor{mg}{\texttt{$binary\_fill\_holes$}} from scipy.ndimage.morphology \cite{noauthor_multi-dimensional_nodate} is used.
An example of this can be seen in \cref{fig:prepro_examples}.

In the coronal view, each slice of the transformed data is originally of shape (63, 48), matching the reference space resolution of \SI{200}{\micro\metre}.
It is then reshaped into (64, 64) by adding a zero padding to the border.

Finally, the images are normalised by first clipping them from the minimum to the \nth{99} percentile of the data in order to remove outliers and then divided by the maximum.

The data set is separated into Training, Validation and Test sets such that 90\% of the total data are used for training and 10\% for testing.
The Validation set is used for the optimisation of hyper parameters while the Test set is used as a measure of extrapolation capability.

\subsection{Data Augmentation} \label{Data Augmentation}

Because of diverse settings in the experiment setup, including animal manipulations causing artifacts, MR image quality can differ substantially between labs and even individual study populations.
To account for these variations, we apply an extensive set of transformations to our data.
This includes rotations of up to 90$^{\circ}$, a width and height shift range of 30 pixels, a shear range of 5 pixels, zoom range of 0.2 and horizontal as well as vertical flips.

This not only increases the data set size but also makes it more representative of the general data distribution of Mice brain MR images and results in a model with a better generalisation capability.

Many more sophisticated methods have been tested, but it has been shown that one of the more successful data augmentation strategies is the simple transformations mentioned above \cite{perez_effectiveness_2017}.


\subsection{Training}
The model was trained slice wise, with the coronal view and 600 as the maximum number of epochs.
The coronal view was chosen over the axial one, because the shapes of the masks are much simpler in the coronal view and thus easier to learn for the network.
Additionally the coronal view has the advantage of higher resolution as the MR images were recorded coronally.

Because of the extensive transformations on the images (see \cref{Data Augmentation}), the augmented data presents much variation.
To alleviate the learning process of the model, we have first trained it on the data without augmentation and then on the augmented data.

To improve the learning process of the network, two callbacks from Keras were used \cite{noauthor_callbacks_nodate}. "ReduceLROnPlateau" reduces the learning rate when the validation loss has stopped improving and "EarlyStopping" stops the training when the validation loss has stopped improving for a number of epochs.
The latter reduces computation time and prevents overfitting.

\subsection{Masking}
In order to improve the SAMRI registration workflow, an additional node is implemented where the images are masked, such that only the brain region remains.
For this process, the image needs to be preprocessed using the operations described in \cref{subsec:Data Set}.
Since the classifier was trained to predict on images of shape (64, 64), the input needs to be reshaped.
If one dimension is bigger than 64, it reshaped into 64 using the function \textcolor{mg}{\texttt{$cv2.resize$}} from the opencv python package. TODO:cite
If one dimension is smaller, a zero padding is added to the border.
For each slice in the image, the classifier then predicts a segmentation of the brain, which is used to create a 3D mask.
The latter is then reshaped into the original shape inverting the preprocessing step, either with the opencv resize method or by cropping.
The workflow then continues with only the Region Of Interest inside the image.

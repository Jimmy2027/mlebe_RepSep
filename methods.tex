\section{Methods}
It is of foremost importance that the complexity of MRI processing interfaces is manageable to their prospective users, such as biologists with only cursory programming experience.
It is, however, also important that the automatable aspect of the interfaces or the sustainability of the software package are not compromized for trivial features which purportedly aid convenience. 
We thus follow  a set of design guidelines, stating that:
each pipeline is represented by a high-level function, whose parameters correspond to operator-understandable concepts (i.e. describe the operations performed, rather than the computational manner in which they are performed); 
pipeline functions are highly parameterized (so that users can change their function to a significant extent without needing to edit the constituent code) but include sane defaults;
graphical or interactive interfaces are to be avoided as they impede reproducibility, encumber the dependency graph, and reduce the sustainability of the project by taxing the development workload without providing usable features to developers.

The language of choice for the pipelining interfaces is Python, owing to its Free and Open Source (FOSS) nature, readability, wealth of available libraries, ease of package management, and its large and dynamic developer community.
While pipeline functions are written in Python, we also provide automatically generated Command Line Interfaces (CLIs), for use directly with Bash.
These autogenerated CLIs ensure that features become available in Bash and Python synchronoulsy, and pipelines behave identically regardless of usage.

Internally, the pipelining functions make use of the Nipype package which provides both directed acyclic graph management and reporting features, as well as jobs, resources, and error reportning functions for the pipeline execution. 

\begin{figure*}
	\includedot[height=37em]{data/generic}
	\includedot[height=39em]{data/legacy}
\end{figure*}

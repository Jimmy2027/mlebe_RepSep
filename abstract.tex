Given the need for inter-subject and inter-study comparability, spatial map registration to a standardized space is an indispensable part of modern Magnetic Resonance Imaging (MRI).
Mouse MRI workflows commonly utilize high-level interfaces specifically intended for human data --- and adapt the data to the interface rather than vice-versa.
Quality control (QC) is commonly performed by interactive operator inspection, making it infrequent, biased, slow, and unreproducible.
In this paper we present a novel registration workflow using the full flexibility of low-level interfaces from one of the most popular normalization toolkits, and provide an optimized set of parameters for mouse brain imaging.
Additionally, we present a QC workflow, which can automatically assess the registration quality of current as well as past processed datasets.
We showcase the capabilities of both workflows, in a comparison of our current workflow and a legacy workflow (the latter illustrating multiple popular hacks - which we detail and comment).

The quality and reliability of scientific results critically depends on reproducible and transparent data processing.
Cross-subject and cross-study comparability of imaging data in general, and magnetic resonance imaging (MRI) data in particular, relies on the quality of registration to a standard reference space.
%%% MRI rather than fMRI
This is not adequately provided by currently used processing workflows, which utilize high-level scripts optimized for human data, and adapt animal data to fit the scripts, rather than vice-versa.
Quality control (QC) in such workflows is operator-interactive, making it infrequent, open to bias, slow, and unreproducible.
In this fully reproducible article we showcase a generic workflow optimized for the mouse brain and a standard reference space suited to harmonize data between analysis and operation.
We present four separate metrics for automated QC, and a visualization method to aid operator inspection.
Benchmarking this workflow against common legacy practices reveals that it performs more consistently, better preserves variance across subjects while minimizing variance across sessions, and improves volume conservation RMSE
\py{pytex_printonly('scripts/vc_rmser.py')}-fold,
and smoothness conservation RMSE
\py{pytex_printonly('scripts/sc_rmser.py')}-fold.
We propose this open architecture and in-built QC as a new standard for small animal MRI registration, ensuring robustness, comparability, and validity of region assignment, important criteria for the comparability of scientific results across centers.
%This workflow sets a new standard for small animal MRI registration, ensuring robustness, comparability, and validity of region assignment, important criteria for the comparability of scientific results across centers.
%%% Stronger statement and less repetition/overt speculation

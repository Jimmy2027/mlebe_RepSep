The reliability of scientific results critically depends on reproducible and transparent data processing.
Cross-subject and cross-study comparability of imaging data in general, and magnetic resonance imaging (MRI) data in particular, is contingent on the quality of registration to a standard reference space.
In small animal MRI this is not adequately provided by currently used processing workflows, which utilize high-level scripts optimized for human data, and adapt animal data to fit the scripts, rather than vice-versa.
In this fully reproducible article we showcase a generic workflow optimized for the mouse brain, alongside a standard reference space suited to harmonize data between analysis and operation.
We present four separate metrics for automated quality control (QC), and a visualization method to aid operator inspection.
Benchmarking this workflow against common legacy practices reveals that it performs more consistently, better preserves variance across subjects while minimizing variance across sessions, and improves both volume and smoothness conservation RMSE approximately
\py{pytex_printonly('scripts/abstract_rmser.py')}.
We propose this open source workflow and the QC metrics as a new standard for small animal MRI registration, ensuring workflow robustness, data comparability, and region assignment validity, important criteria for the comparability of scientific results across experiments and centers.

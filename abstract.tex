Given the need for inter-subject and inter-study comparability, spatial map registration to a standardized space is an indispensable part of modern Magnetic Resonance Imaging (MRI).
Human MRI research has produced numerous registration toolkits and associated workflow implementations, predominantly accessed via high-level interfaces contain hard-coded parameters optimized for specific human MRI use cases.
Animal MRI commonly makes use of these high-level interfaces, and implements additional hacks to mitigate the nonhuman idiosyncracies of the species being imaged --- instead of optimizing the workflow for the data at hand.
Quality control is commonly performed by operator inspection, making it infrequent, biased, slow, and unreproducible.
In this paper we present a novel workflow using the full flexibility of low-level functions from one of the most popular neuroimaging registration toolkits, and provide an optimized set of parameters for small animal imaging.
Additionally, we present a quality control (QC) workflow, which can automatically assess the registration quality of processed datasets.
We showcase the capabilities of both workflow, by comparing our current registration performance with that of a legacy registration workflow (containing multiple popular hacks - which we specifically critique).

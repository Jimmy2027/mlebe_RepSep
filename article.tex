\documentclass[10pt,a4paper,twocolumn,english]{article}

\title{An Optimized Registration Pipeline and Standard Geometric Space for Small Animal Brain Imaging}
\author{
	\authorstyle{Horea-Ioan Ioanas\textsuperscript{1}}
	\authorstyle{Markus Marks\textsuperscript{2}}
	\authorstyle{Mehmet Fatih Yanik \textsuperscript{2}}
	\authorstyle{Markus Rudin\textsuperscript{1}}
	\newline
	\textsuperscript{1}\institution{Institute for Biomedical Engineering, ETH and University of Zurich}\\
	\textsuperscript{2}\institution{Institute of Neuroinformatics, ETH and University of Zurich}
}
\date{}
\usepackage[dvipsnames]{xcolor}
\usepackage[utf8]{inputenc}
\usepackage[T1]{fontenc}
\usepackage{geometry}
	\geometry{
		top=1cm,
		bottom=1.5cm,
		left=2cm,
		right=2cm,
		includehead,
		includefoot,
	}
\usepackage[hidelinks]{hyperref}
\usepackage{graphicx}
\usepackage{lastpage}
\usepackage[iso]{isodate}
\usepackage{booktabs}


%VARIABLES
%set up a more stable buffers (the normal ones gets blanked after the titlepage is invoked)
\makeatletter
\let\mytitle\@title
\let\myauthor\@author
\let\mydate\@date
\makeatother


%HEADER/FOOTER
\usepackage{fancyhdr}
	\pagestyle{fancy}
	\renewcommand{\headrulewidth}{0.0pt}
	\renewcommand{\footrulewidth}{1pt}
	\renewcommand{\sectionmark}[1]{\markboth{#1}{}}

	\lhead{}
	\chead{\small\usefont{T1}{phv}{m}{n}\textcolor{Gray}{\mytitle}}
	\rhead{}

	\lfoot{}
	\cfoot{\footnotesize\usefont{T1}{phv}{m}{n} \textcolor{Gray}{\today}}
	\rfoot{\footnotesize\usefont{T1}{phv}{m}{n} \textcolor{Gray}{Page} \thepage\ \textcolor{Gray}{of \pageref{LastPage}}}

	\fancypagestyle{firstpage}{
   		\fancyhf{}
		\lfoot{}
		\cfoot{\footnotesize \usefont{T1}{phv}{m}{n}\textcolor{Gray}{\today}}
		\rfoot{\footnotesize \usefont{T1}{phv}{m}{n}\textcolor{Gray}{Page} \thepage\ \textcolor{Gray}{of \pageref{LastPage}}}
	}

\usepackage{etoolbox}
	\makeatletter
		\patchcmd{\footrule}{\hrule}{\color{Gray}\hrule}{}{}
	\makeatother


%TITLE
\newcommand{\authorstyle}[1]{{\large\usefont{T1}{phv}{b}{n}\color{Gray}#1}}
\newcommand{\institution}[1]{{\footnotesize\usefont{T1}{phv}{m}{n}\color{Black}#1}}
\usepackage{titling}
\newcommand{\HorRule}{\color{Gray}\rule{\linewidth}{1pt}}
\pretitle{
	\vspace{-6em} %move title section up
	\HorRule %horizontal rule abve title
	\vspace{1.2em}
	\huge\usefont{T1}{phv}{b}{n}
	\color{Black} %title color
}
\posttitle{\par\vskip 0.5em}
\preauthor{}
\postauthor{
	\vspace{1.2em}
	\par\HorRule %horizontal rule below title
	\vspace{-1em}
}


%SECTIONS
%make sure the first paragraph of a section is not indented
\makeatletter
	\let\@afterindenttrue\@afterindentfalse
	\@afterindentfalse
\makeatother

\usepackage[explicit]{titlesec}
\titleformat{\section}[hang]{\large\usefont{T1}{phv}{b}{n}\color{Black}}{}{0em}{#1}%
\titlespacing{\section}{0em}{1.5em}{0.4em}
\titleformat{\subsection}[hang]{\usefont{T1}{phv}{b}{n}\color{Black}}{}{0em}{#1}%
\titlespacing{\subsection}{0em}{1em}{0.2em}
\titleformat{\subsubsection}[hang]{\footnotesize\usefont{T1}{phv}{b}{n}\color{Black}}{}{0em}{#1}%
\titlespacing{\subsubsection}{0em}{0.75em}{0em}

%PYTHONTEX
\usepackage[autoprint=false, gobble=auto, pyfuture=all]{pythontex} %create figures on-line directly from python!
\usepackage{pgf} %we need this backend

\input{pythontex/functions.tex}
\begin{pythontexcustomcode}[begin]{py}
DOC_STYLE = 'article/main.conf'
pytex.add_dependencies(DOC_STYLE, 'article/3dplot.conf', 'article/bsc_percentage.conf')
\end{pythontexcustomcode}



%BODY
\setlength{\columnsep}{7mm}


%BIBLIOGRAPHY
%\usepackage[backend=bibtex,style=authoryear,natbib=true]{biblatex}
%\addbibresource{bib.bib}


\begin{document}
\maketitle
\thispagestyle{firstpage}
\noindent
\textsf{%
\textbf{Abstract ---}
	%%briefly explain why
%%  - extraction ROI 
%%  - registration into a standard reference space
As part of the analysis of high-field mouse MRI data, relevant brain tissue needs to be selected via a mask.
For this process to be performed automatically, brain voxels need to be classified based both on their signal intensity and position in the image.
Nowadays, deep neural networks are the state-of-the-art methods for tissue segmentation in biomedical imaging, and thus constitute a promising method for preclinical neuroscience.
%% outperforms current methods is a pretty high bar. We haven't really compared it to e.g. BET and other heuristic human-MRI-derived classifiers. You can still make the case that it is better, but that's a case you need to make on theoretical grounds unless you have data for the comparison.
We present a deep learning enabled framework for segmentation of brain tissue in functional and structural MR images, \textit{that outperforms \textbf{current methods}, requiring only a fraction of the processing time needed by them}. \todo{not true}

}

\section{Background}

In order to make any generalizable statements regarding brain function and organization, an equivalence between brain areas across individuals needs to be established.
This is most commonly done by spatial (rigid, affine, and non-linear) transformation of all brain maps in a study to a population or atlas template.
This process, called registration, is consequently performed as part of any neuroimaging workflow attempting to produce results which are both spatially resolved and generalizable across the population.

The computations required for registration are commonly performed at the very onset of the preprocessing workflow (possibly after slice-timing correction),
though --- depending on the workflow --- the actual image manipulation may only take place much later, once inter-subject comparison is needed [[[paper which does this and says that it minimizes interpolation]]].
As a consequence of this marginal positioning in the preprocessing sequence, as well as its general independence from experimental designs and hypotheses, registration is often relegated to default values and exempt from rigorous design efforts and QC.

QC is a notable issue for human as well as mouse brain imaging;
and is not limited by a lack of reporting functions, but by a lack of reporting metrics which can easily be communicated for a population.
Concerning the registration itself, human brain imaging uniquely benefits from high-level functions (shipped by most software packages providing registration functionality), which are optimized for the size and spatial features of the human brain.
The availability and widespread use of such functions greatly mitigates the issue that users would not themselves implement a rigorous registration workflow.
In mouse brain imaging, however, registration is frequently performed using the selfsame high-level functions from human brain imaging, and applying a series of manipulations which adjust the nature of mouse brain data to fit human brain priors and parameters optimized for human brain registration.
This general approach entails numerous issues, and represents a notable hurdle in the way of improving mouse brain imaging, yet needs to be detailed as a current common practice.

\subsection{Manipulations}
The foremost data manipulation procedure is the adjustment of the voxel dimensions (as recorded in the NIfTI header \cite{nifti}), so that the mouse data represent a volume corresponding to what human-optimized brain extraction, bias correction, and registration interfaces expect (commonly this constitutes a 10-fold increase in each dimension).

Another notable data manipulation procedure consists in adjusting the data matrix content so that human-prior based brain extraction will produce acceptable results.
While conceptually superior solution (e.g. adapting parameters and priors to animal data \cite{rbet,Oguz2014}) are available and might remove the need for data adaptation at this step, rudimentary solutions remain popular.
Many consist of applying an empirically determined percentile threshold, intended to clear non-brain or distal brain tissue by intensity, and to leave a more spherical brain for the human masking function to operate on.
Notably, both the function adaptations for animal data and the animal data matrix adaptations for use with human brain extraction functions are known to wholly or partly remove the olfactory bulbs (if at all present in the acquired data) --- which is why sometimes the choice is made to instead simply forego brain extraction.

Commonly, the orientation of the data is seen as problematic, and consequently deleted.
This procedure consists in resetting the S-Form affine from the NIfTI header to zeroes, and is intended to mitigate incorrect data orientation produced by the scanner.
While it is true that the scanner affine reported for mouse data may be nonstandard (the confusion is two-fold: mice lie prone with the coronal plane progressing axially whereas higher primates lie supine with the horizontal plane progressing axially), it is equally true that affines of mouse brain templates may be nonstandard.
A different but related manipulation is dimension swapping, which changes the order of the NIfTI data matirx rather than the affine.
Occasionally, correct or automatically redressable affines are thus deleted in order to correspond to a malformed template.

\subsection{Templates}
As the above eminently demonstrates, the template is a key component of a registration workflow.
Templates used for mouse brain MRI registration are highly heterogeneous, and include histological templates, as well as ex vivo MRI templates, scanned either inside the intact skull or after physical brain extraction.

Histological templates benefit from higher resolution and access to molecular imaging data in the same coordinate space.
Such histological templates are however not produced in volumetric sampling analogous to MRI, and are often not assigned a meaningful affine after conversion to NIfTI.
Histological contrast may only poorly correlate with any MR contrast, such as $T_1$, $T_2$, or $T_2^*$, making registration --- especially at lower resolutions --- less reliable, or necessitating the use of similarity metrics which impose additional restrictions.
Not least of all, histological templates may be severely deformed (and may lack distal parts of the brain such as the olfactory bulbs) due to the brain extraction and sampling process;
and data registered to them may be particularly difficult to use for navigation in the intact mouse brain, e.g. during data acquisition or stereotactic surgery,

Ex-vivo templates based on extracted brains share most of the deformation issues present in histological templates;
they are, however available in native MR contrasts, commonly $T_2$, making registration far easier.
They suffer, in comparison, from a lower resolution, and would need to have any histological data from a histological atlas space first registered to them.
Ex-vivo templates based on intact mouse heads provide both native MR contrasts and brains free of deformation and supporting whole brain sampling.
They share the resolution issues and the need to first register histological data, which is also seen in ex-vivo templates based on extracted brains.

The registration process consists of the computation of a transformation matrix (which can have multiple layers of complexity, e.g. combining a transformation from the functional to the structural scan with a transformation from the structural scan to a general template) and the application of the transformation matrix in s process called warping.

\subsection{Challenges}
It thus becomes obvious that in addition to the issues shared with registration of human MRI data, the current challenges in mouse MRI registration consist in minimizing workarounds and reliance on high-level interfaces with inappropriate parameters and priors, as well as the reduction (perhaps through merger) of standard space templates.
Information removal during preprocessing is a notable issue, since data lost at the onset of the neouroimaging workflow, will persist in all downstream steps and preclude numerous modes of analysis.


\section{Methods}\label{sec:methods}
For the benchmarking of the two workflows, the same methods that are described in the original paper have been applied in this work.
A more detailed description can be found there.

\begin{sansmath}
\py{pytex_subfigs(
        [
                {'script':'scripts/classifier/uprepex.py', 'label':'exunprepro', 'conf':'article/1col.conf', 'options_pre':'{.48\\textwidth} \\vspace{-2em}',
                        'options_pre_caption':'\\vspace{0.1em}',
                        'options_post':'\\vspace{1em}',
                        'caption':'Example of an unpreprocessed slice.'
                        ,},
                {'script':'scripts/classifier/prepex.py', 'label':'exprepro', 'conf':'article/1col.conf', 'options_pre':'{.48\\textwidth} \\vspace{-2em}',
                        'options_pre_caption':'\\vspace{0.1em}',
                        'options_post':'\\vspace{1em}',
                        'caption':'Example of a preprocessed slice.'
                        ,},
                ],
        caption='\\textbf{The preprocessing removes the mask there, where the image-pixelvalues are 0.}\\
        Plots of the same image, superposed with the template mask, with and without preprocessing.
        ',
        label='fig:prepro_examples',
        )}
\end{sansmath}

\begin{sansmath}
\py{pytex_fig('scripts/classifier/plt_trainset.py',
        conf='article/2*8plot.conf',
        label='trainset',
        caption='
        Augmented samples from the Training set.
        ',
        multicol=True,
        )}
\end{sansmath}

\subsection{Model}
As the architecture of the classifier, the U-Net from Ronneberger et al \cite{ronneberger_u-net:_2015} was chosen based on its high performance in the field of biomedical image segmentation.
This is a convolutional neural network that consists of a contracting path that captures context in addition to a symmetric expanding path that enables precise localization.
Localization in this context means that a class label is assigned to each pixel.
We used the U-Net implementation from zhixuhao \cite{zhixuhao_zhixuhao/unet_2020}, written in Keras.
Keras is a high-level neural networks API, written in Python and capable of running on top of TensorFlow, CNTK, or Theano.
It allows for easily readable code and thus makes the workflow easier to reproduce.

The implementation of the U-Net from zhixuhao has, in addition to the original architecture, two drop-out layers.
A drop-out layer randomly sets a fraction of input units from the layer to 0 at each update during training time.
The set fraction rate is 0.5.
It is known that dropout helps prevent overfitting and greatly improves the performance of deep learning models \cite{srivastava2014dropout}.

The model was trained using the Dice loss, which is computed from the Dice score.
It calculates the similarity of two binary samples X and Y with
\begin{equation}\label{eqDcoef}
D_{coef} = \frac{2|X\cap Y|}{|X|+|Y|}
\end{equation}

It is a quantity ranging from 0 to 1 that is to be maximized.
The loss is then calculated with $1-D_{coef}$.
Because the Dice loss is not differentiable, small changes need to be made.
In our case, the two samples to be compared are normalized, grey valued images and are thus not binary but have values between 0 and 1.
Additionally, instead of using the logical operation \textit{and}, element wise products are used to approximate the non-differentiable intersection operation.
To avoid a division by zero, $+1$ is added on the numerator and denominator.

Because many more pixels in the masks are 0 than 1, there is a class imbalance problem.
This is a problem because in this case a false positive gives a much higher loss than a false negative.
For example, predicting only black would give an acceptable loss, while predicting only white pixels would not.
Using the Dice coefficient as a loss function for training should make it invariant to this class imbalance problem as stated by Fausto Milletari et al. in \cite{milletari_v-net:_2016}.

\subsection{Data Set} \label{subsec:Data Set}
The data set consists of 3D MR images taken from an aggregation of three studies: irsabi \cite{irsabi_bidsdata}, opfvta \cite{ioanas_whole-brain_nodate}, drlfom \cite{ioanas_effects_nodate} and other unpublished data, acquired with similar parameters.

The measured animals were fitted with an optic fiber implant ($\mathrm{l=\SI{3.2}{\milli\meter} \ d=\SI{400}{\micro\meter}}$) targeting the Dorsal Raphe (DR) nucleus in the brain stem.
Using this dataset shows that the classifier is robust to these types of experiment setups.

Images from the irsabi study are only used for quality control of the registration and are thus unknown to the classifier.
It is the same dataset that was used to benchmark the Generic workflow in the original paper and thus allows for a better estimation of the general performance of our improved pipeline.

The images are transformed into a standard space using a template mask via SAMRI \cite{noauthor_ibt-fmi/samri_2019} and are thus defined in the same affine space.
SAMRI is a data analysis package of the ETH/UZH Institute for Biomedical Engineering.
It is equipped with an optimized registration workflow and standard geometric space for small animal brain imaging \cite{ioanas_optimized_2019}.

Because of variance in mouse brain anatomy and in the experiment setup, some of the transformed data do not overlap perfectly with the reference template.
To filter these images out, most of the incongruent slices were removed manually from the data set.

For the registration of the images, a padding was needed to make the originally not affine space affine.
As a result, the 3D volumes present many zero-valued slices, some of them overlapping with the mask.

Since it is not wanted for the model to predict a mask on black slices, the mask is set to zero where the image is zero-valued.
Because some pixels representing the brain tissue are zero-valued, holes result from this operation.
To patch these, the function \textcolor{mg}{\texttt{$binary\_fill\_holes$}} from scipy.ndimage.morphology \cite{noauthor_multi-dimensional_nodate} is used.
An example of the preprocessing can be seen in \cref{fig:prepro_examples}.

In the coronal view, each slice of the transformed data is originally of shape (63, 48), matching the reference space resolution of \SI{200}{\micro\metre}.
It is then reshaped into (128, 128) by first zero-padding the smaller dimension to the same size as the bigger one and then reshaping the image into 128 using the function \textcolor{mg}{\texttt{$cv2.resize$}} from the opencv python package \cite{noauthor_opencv-python_nodate}.

Finally, the images are normalized by first clipping them from the minimum to the \nth{99} percentile of the data to remove outliers and then divided by the maximum.

The data set is separated into Training, Validation and Test sets such that 90\% of the total data are used for training and validation while 10\% are used for testing.
This is done with the help of the function \textcolor{mg}{\texttt{$train\_test\_split$}} from the package sklearn.model$\_$selection \cite{scikit-learn}.
The Validation set is used for the optimization of hyperparameters while the Test set is used as a measure of extrapolation capability.

\subsection{Data Augmentation} \label{Data Augmentation}

Because of diverse settings in the experiment setup, including animal manipulations causing artifacts, MR image quality can differ substantially between labs and even individual study populations.
To account for these variations, we apply an extensive set of transformations to our data.
This includes rotations of up to 90$^{\circ}$, a width and height shift range of 30 pixels, a shear range of 0.5 pixels, zoom range of 0.3, brightness range of (0.7, 1.3) and horizontal as well as vertical flips.
Additionally a gaussian noise with a variance range of (0, 0.001) is added to the image.

This not only increases the data set size but also makes it more representative of the general data distribution of mice brain MR images and results in a model with a better generalization capability.

Many more sophisticated methods have been tested, but it has been shown that one of the more successful data augmentation strategies is the simple transformations mentioned above \cite{perez_effectiveness_2017}.


\subsection{Training}
The model was trained slice wise, with the coronal view and 600 as the maximum number of epochs.
The coronal view was chosen over the axial one, because the shapes of the masks are much simpler in the coronal view and thus easier to learn for the network.

Additionally, the coronal view has the advantage of higher resolution as the MR images were recorded coronally.

To improve the learning process of the network, two callbacks from Keras were used \cite{noauthor_callbacks_nodate}.
"\textcolor{mg}{\texttt{$ReduceLROnPlateau$}}" reduces the learning rate when the validation loss has stopped improving and "\textcolor{mg}{\texttt{$EarlyStopping$}}" stops the training when the validation loss has stopped improving for a number of epochs.
The latter reduces computation time and prevents overfitting.

\subsection{Masking}
To improve the SAMRI registration workflow, an additional node is implemented where the images are masked, such that only the brain region remains.
To alleviate the task of the classifier, the image is first bias-corrected using the "\textcolor{mg}{\texttt{$N4BiasFieldCorrection$}}" function of the ANTs package, with spatial parameters used in the samri functions.
The image is then resampled into the resolution of the template space, which has a voxel size of $0.2\times 0.2 \times 0.2$.
This is done with the \textcolor{mg}{\texttt{$Resample$}} command from the FSL library which is an analysis tool for FMRI, MRI and DTI brain imaging data \cite{fsl}.
Then, the image is preprocessed using the operations described in \cref{subsec:Data Set}.
Since the classifier was trained to predict on images of shape (128, 128), the input needs to be reshaped.
The slice-wise predictions of the model are reconstructed to a 3D mask via the command \textit{Nifit1Image} from the neuroimaging python package nibabel \cite{noauthor_neuroimaging_nodate}.
This is done using the same affine space as the input image.
The latter is then reshaped into the original shape inverting the preprocessing step, either with the opencv resize method or by cropping.
Additionally, the binary mask is resampled into its original affine space, before being multiplied with the brain image to extract the ROI.
The workflow then continues with only the Region Of Interest as the image.

\subsection{Metrics}

The VCF uses the 66\textsuperscript{th} voxel intensity percentile of the raw scan before any processing as definition of the brain volume.
The VCF is then obtained with \cref{eq:vcf}, where $v$ is the voxel volume in the original space, $v^\prime$ the voxel volume in the transformed space, $n$ the number of voxels in the original space, $m$ the number of voxels in the transformed space, $s$ a voxel value sampled from the vector $S$ containing all values in the original data, and $s^\prime$ a voxel value sampled from the transformed data.

\begin{equation} \label{eq:vcf}
        V\!C\!F
        = \frac{v^\prime\sum_{i=1}^m [s^\prime_i \geq P_{66}(S)]}{v\sum_{i=1}^n [s_i \geq P_{66}(S)]}
        = \frac{v^\prime\sum_{i=1}^m [s^\prime_i \geq P_{66}(S)]}{v \lceil0.66n\rceil}
\end{equation}

The bootstrapped distribution of the RMSE (\cref{eq:RMSE}) is obtained by resampling the VCF distributions 10000 times with replacement, and computing the RMSE for every sample.

\begin{equation} \label{eq:RMSE}
        RMSE = \sqrt{(\text{mean}((1 - \text{VCF})^2)}
\end{equation}

The SCF metric is based on the ratio of smoothness before and after processing.
It is obtained via \cref{eq:acf}, where $r$ is the distance of two amplitude distribution samples, $a$ is the relative weight of the Gaussian term in the model, $b$ is the width of the Gaussian and $c$ the decay of the mono-exponential term \cite{cox2017fmri}.

\begin{equation} \label{eq:acf}
        ACF(r)
        = a * e^{ -r^{2}/ (2 * b^{2}) } + (1 - a) + e^{-r/c}
\end{equation}

The for the MS relevant statistical power is obtained via the negative logarithm of first-level p-value maps.
Voxelwise statistical estimates for the probability that a time course could --- by chance alone --- be at least as well correlated with the stimulation regressor as the voxel time course measured are averaged via \cref{eq:ms}, where $n$ represents the number of statistical estimates in the scan, and $p$ is a p-value.

\begin{equation} \label{eq:ms}
        M\!S = \frac{\sum_{i=1}^n -log(p_i)}{n}
\end{equation}

\section{Data and Code Availability}

%The data archive relevant for this article is freely available \cite{mlebe_bidsdata}, and automatically accessible via the Gentoo Linux package manager.
In addition to the workflow code \cite{mlebe, samri}, we openly release the re-executable source code \cite{mlebe_repsep} for all statistics and figures in this document.
The herein introduced novel method as well as the benchmarking are thus fully transparent and reusable for further data.
\section{Evaluation}
For the quality control of the workflow, we first evaluate the classification process, followed by a benchmark between the Generic and the improved "Masked" workflow.
Statistics for the Volume Conservation and the Smoothness Conservation are presented with respect to the distributions of the absolute distances to the optimal value 1.

\subsection{Classification}
%%% Again, background or discussion
Quality control of our classifier is difficult in the sense that the template mask does not always overlap perfectly with the brain region, such that small deviances of the predictions compared to the template could actually be caused by the prediction being more accurate than the template.
Nevertheless, it is useful to verify whether the output is similar to the template, as it should be.
As a similarity metric between the template mask and the classifier output we have used the Dice score (see \cref{eqDcoef}).
The average Dice score on the test data set is $D_{coef}= $ \py{boilerplate.print_dice()} $\sim 1$, indicating that classifier output has only minor changes in comparison with the template.
% todo should I give more scores? AUC, ...

\begin{sansmath}
\py{pytex_fig('scripts/classifier/plt_testset_examples.py',
        conf='article/wide.conf',
        label='testset_ex',
        caption='
                \\textbf{The Classifier predicts a similar mask to the ground truth.}
                Randomly picked plots from the test set illustrate the predictions of the classifier.
                The first row presents the input image, the second the ground truth and the third row shows the predictions of the classifier.
                ',
        multicol=True,
        )}
\end{sansmath}

\subsection{Workflow}
We use an established palette of workflow evaluation metrics --- inspecting volume and smoothness conservation, as well as downstream effects on basic functional analysis \cite{ioanas_optimized_2019} --- to benchmark the novel SAMRI Masked workflow against the SAMRI Generic workflow.

\subsection{Volume Conservation}

\begin{sansmath}
\py{pytex_subfigs(
        [
                {'script':'scripts/vcc_violin.py', 'label':'vccv','conf':'article/1col.conf', 'options_pre':'{.48\\textwidth}',
                        'options_pre_caption':'\\vspace{-1.5em}\\',
                        'options_post':'\\vspace{1em}',
                        'caption':'Comparison across workflows and functional contrasts.'
                        ,},
                {'script':'scripts/scf_violin_contrasts.py', 'label':'sccv','conf':'article/1col.conf', 'options_pre':'{.48\\textwidth}',
                        'options_pre_caption':'\\vspace{-1.5em}\\',
                        'options_post':'\\vspace{1em}',
                        'caption':'Comparison across workflows and functional contrasts.'
                        ,},
                {'script':'scripts/vc_violin_absdiff.py', 'label':'vcfb','conf':'article/1col.conf', 'options_pre':'{.48\\textwidth}',
                        'options_pre_caption':'\\vspace{-1.5em}\\',
                        'options_post':'\\vspace{1em}',
                        'caption':'Comparison of the distributions of the absolute distances to 1, across workflows and functional contrasts.'
                        ,},
                {'script':'scripts/scf_violin_absdiff.py', 'label':'scfb','conf':'article/1col.conf', 'options_pre':'{.48\\textwidth}',
                        'options_pre_caption':'\\vspace{-1.5em}\\',
                        'options_post':'\\vspace{1em}',
                        'caption':'Comparison of the distributions of the absolute distances to 1, across workflows and functional contrasts.'
                        ,},
                ],
        caption='\\textbf{Both the SAMRI Generic and the Masked workflow optimally and reliably conserve volume and smoothness, the latter showing values that are closely distributed to 1.}
        Plots showing the distribution of two target metrics in the first row, together with the respective distributions of the absolute distances to 1 in the second row. Solid lines in the colored distribution densities indicate the sample mean and dashed lines the inner quartiles.
        ',
        label='fig:vc',
        )}
\end{sansmath}

Volume Conservation Factor (VCF) \cite{ioanas_optimized_2019} measures the registration induced deformation of the scanned brain, by computing the ratio of the brain volume before and after preprocessing.
A positive ratio indicates that the brain was stretched to fill the template space, while a negative ratio indicates that non-brain voxels were introduced in the template brain space.
Volume conservation is highest for a VCF equal to 1, indicating that the preprocessing has no influence on the brain volume of the scans.

As seen in \cref{fig:vcfb}, we note that in the described dataset the absolute distance of the VCF to 1 is sensitive to the workflow
(\py{boilerplate.fstatistic('Processing', dependent_variable='Abs(1 - Vcf)', expression='Processing+Contrast', condensed=True)}).
The performance of the Generic SAMRI workflow is different from that of the Masked, yielding a two-tailed p-value of \py{pytex_printonly('scripts/vc_t.py')}.
With respect to the data break-up by contrast (CBV versus BOLD, \cref{fig:vccv}), we see no notable main effect for the contrast variable
(VCF of \py{boilerplate.corecomparison_factorci('Contrast[T.CBV]',dependent_variable='Abs(1 - Vcf)', expression='Processing+Contrast')}).
%%% Ok, you introduce the VCF and you show the VCF plots primarily, and then you list the stats for RMSE only? You should definitely write out the modelling evaluation for VCF. My recommendation is to write down a paragraph for the VCF evaluation, and then RMSE evaluation, plot-wise you show a composite figure with subfigures for VCF, SCF, and the respective bootstrapped RMSE.

%%% briefly introduce what the RMSE tells you here.
We note that there is a significant variance decrease in all conditions for the Masked workflow
(\py{boilerplate.varianceratio()}-fold).
Further, we note that the root mean squared error ratio favours the Masked workflow
($\mathrm{RMSE_{M}/RMSE_{G}\simeq} \py{pytex_printonly('scripts/vc_rmser.py')}$).

\subsection{Smoothness Conservation}

%%% Same comments as for VCF

Smoothing is a popular tool employed by many preprocessing functions to increase the signal-to-noise ratio.
Image smoothness comes at the cost of image contrast as well as feature saliency and has been shown to result in inferior anatomical alignment \cite{fmriprep} due to the loss of spatial resolution.
As an indicator of image smothness induced by the workflow, the Smoothness Conservation Factor (SCF) \cite{ioanas_optimized_2019} expresses the ratio between the smoothness of the preprocessed images and the smoothness of the original images.
Smoothess Conservation is highest for a SCF equal to 1, indicating that the preprocessing does not influence image smoothness.

While the performance of the Generic SAMRI workflow is only slightly different from that of the Masked workflow, the root mean squared error ratio favors the Masked workflow ($\mathrm{RMSE_{M}/RMSE_{G}\simeq} \py{pytex_printonly('scripts/scf_rmser.py')}$).

Descriptively, we observe that neither the Generic nor the Masked workflow introduce a strong smoothing (SCF of \py{boilerplate.factorci('Processing[T.Masked]', df_path='data/smoothness.csv',dependent_variable='Abs(1 - Scf)', expression='Processing+Contrast')}).

Further, we note that there is a slight variance decrease for the Masked workflow
(\py{boilerplate.varianceratio(df_path='data/smoothness.csv',dependent_variable='Smoothness Conservation Factor', max_len=3)}
-fold).

Given the break-up by contrast shown in \cref{fig:sccv}, we see no effect for the contrast variable
(SCF of \py{boilerplate.corecomparison_factorci('Contrast[T.CBV]', df_path='data/smoothness.csv', dependent_variable='Abs(1 - Scf)', expression='Processing+Contrast')}).

\subsection{Functional Analysis}

%%% Same comments as for VCF, these three sections can basically be copy-pasted in as far as the sentence structure is concerned.
Functional Analysis expresses the significance detected across all voxels of a scan by computing the Mean Significance (MS) \cite{ioanas_optimized_2019}.

We observe that the Masked level of the workflow variable does not introduce a notable significance loss
(MS of \py{boilerplate.factorci('Processing[T.Masked]', df_path='data/functional_significance.csv', dependent_variable='Mean Significance')}).
Furthermore, we note a slight variance decrease in all conditions for the Masked workflow
(\py{boilerplate.varianceratio(df_path='data/functional_significance.csv', dependent_variable='Mean Significance')}-fold).

With respect to the data break-up by contrast (\cref{fig:mscv}), we see no notable main effect for the contrast variable
(MS of \py{boilerplate.corecomparison_factorci('Contrast[T.CBV]', df_path='data/functional_significance.csv', dependent_variable='Mean Significance')}).
%and no notable effect for the contrast-template interaction
%(MS of \py{boilerplate.corecomparison_factorci('Processing[T.Legacy]:Contrast[T.CBV]', df_path='data/functional_significance.csv', dependent_variable='Mean Significance')}).
%
%Functional analysis effects can further be inspected by visualizing the statistic maps.
%Second-level t-statistic maps depicting the CBV and BOLD omnibus contrasts (common to all subjects and sessions) provide a succinct overview capturing both amplitude and directionality of the signal (\cref{fig:m}).
%While the most salient feature of this figure is the qualitative distribution difference between CBV and BOLD scans, we note that this is to be expected given the different nature of the processes, and is wholly orthogonal to the question of registration.
%The differential coverage is crucial to the examination of registration quality and its effects on functional read-outs.
%We note that processing with the Generic* workflow (\cref{fig:mllc,fig:mllb}), does not induce issues with statistic coverage alignment and overflow.

\subsection{Variance Analysis}

\begin{sansmath}
\py{pytex_fig('scripts/variance_catplot.py',
        conf='article/varplot.conf',
        label='var',
        caption='
                \\textbf{Both the Generic and the Masked workflow minimize trial-to-trial variability while conserving subject-wise variability.}
                Swarmplots of three metric scores illustrate similarity of preprocessed images for the two corresponding workflow templates, plotted across subjects (separated into x-axis bins) and sessions (individual points in each x-axis bin), for the CBV contrast.
                ',
        multicol=True,
        )}
\end{sansmath}

As an additional metric for the comparison between workflows, we evaluate if physiological meaningfull variability is retained across repeated measurements.
It is based on the assumption that adult mouse brains retain size, shape, and implant position in the absence of intervention, throughout the 8 week study period \cite{ioanas_optimized_2019}.
Examining the similarity between the template and preprocessed scans, session-wise variability should be smaller than subject-wise variability.
This comparison is performed using a type 3 ANOVA, modeling both the subject and the session variables.
For this assessment three metrics are used, with maximal sensitivity to different features:
Neighborhood Cross Correlation (CC, sensitive to localized correlation),
Global Correlation (GC, sensitive to whole-image correlation),
and Mutual Information (MI, sensitive to whole-image information similarity).

\cref{fig:var} renders the similarity metric scores for both the SAMRI Generic and Masked workflows.
Both, the Generic and the Masked workflow produce results which show a higher F-statistic for the subject than for the session variable.
For the Masked workflow, F-statistics show:
CC (subject: \py{boilerplate.variance_test('C(Subject)','Masked','CC', condensed=True)}, session: \py{boilerplate.variance_test('C(Session)','Masked','CC', condensed=True)}),
GC (subject: \py{boilerplate.variance_test('C(Subject)','Masked','GC', condensed=True)}, session: \py{boilerplate.variance_test('C(Session)','Masked','GC', condensed=True)}),
and MI (subject: \py{boilerplate.variance_test('C(Subject)','Masked','MI', condensed=True)}, session: \py{boilerplate.variance_test('C(Session)','Masked','MI', condensed=True)}).

For the Generic SAMRI workflow, resulting data F-statistics show:
CC (subject: \py{boilerplate.variance_test('C(Subject)','Generic','CC', condensed=True)}, session: \py{boilerplate.variance_test('C(Session)','Generic','CC', condensed=True)}),
GC (subject: \py{boilerplate.variance_test('C(Subject)','Generic','GC', condensed=True)}, session: \py{boilerplate.variance_test('C(Session)','Generic','GC', condensed=True)}),
and MI (subject: \py{boilerplate.variance_test('C(Subject)','Generic','MI', condensed=True)}, session: \py{boilerplate.variance_test('C(Session)','Generic','MI', condensed=True)}).

\section{Optimization}

\section{3D Plotting}
	In this section we demonstrate 3D plotting performance for this document class.

	Matplotlib's 3D plots are associated with a number of intricacies (such as axes.facecolor specifying the background of the 2D output figure and not the 3D grid --- which is why they are particularly interesting. 
	
	\py{pytex_fig('scripts/3dplot.py', conf='article/3dplot.conf', label='3dplot', caption='A 3D plot.')}

\section{Other Plots}
	Here we demonstrate more generic plotting capabilities.

	\py{pytex_fig('scripts/bsc_percentage.py', conf='article/bsc_percentage.conf', label='bsc_percentage', caption='Percentage of Bachelor’s degrees conferred to women in the U.S.A. by major (1970-2011).', multicol=True)}

\section{Discussion}

We conclude from the evaluation that the workflow and template design presented herein offer significant advantages in terms of reducing coverage overestimation, and effective loss of resolution.
This is most clearly highlighted by Volume Conservation (\cref{fig:vc}), Smoothness Conservation (\cref{fig:sc}), and Variance Analysis (\cref{fig:var}), where in all cases the joint usage of the SAMRI Generic workflow and Generic template outperforms all other combinations of the multivariate analysis.
This spatial robustness is also revealed in a qualitative examination of higher-level functional maps (\cref{fig:m}), where only the combination of the Generic workflow and the Generic template provides accurate coverage for both BOLD and CBV modalities.
These benefits are provided without compromising statistical power (\cref{fig:ms}), and this also holds for both CBV and BOLD contrasts (\cref{fig:vccv,fig:mscv}).
Additionally, the performance of the Generic processing workflow is more consistent, as shown in notable reductions of the standard deviation for both VCS, SCF, and MS (\cref{fig:vcv,fig:msv,fig:scv}).

These features are augmented by design benefits such as added transparency and parameterization of the workflow (which can more easily be inspected and further improved or customized by the end user), veracity of resulting data headers, and spatial coordinates more meaningful for surgery and histology.

We acknowledge that our recommendation can only extend as far as favouring the SAMRI Generic workflow as an indivisible unit.
More fine-grained processing steps within the workflow (e.g. the inclusion or exclusion of a skull stripping step) cannot be reliably advocated or dismissed based solely on the holistic evaluation presented here.
Further, we hold that the two workflows discussed represent discrete processing principles and are not arbitrarily mixable (e.g. intensity manipulations go hand in hand with the masking choice, and the structural intermediary goes hand in hand with the registration interface choice).
Thus, in support of \textit{individual} nodes in the processing workflows (\cref{fig:wfg}), we only make theoretical claims.

\subsection{Quality Control}

We note that a major contribution of this work is the implementation of multiple metrics for simple, powerful and robust Quality Control for registration performance (VCF, SCF, and Variance Analysis) and the release of a dataset suitable for such multifaceted benchmarking.
The VCF provides a good quantitative estimate of distortion prevalence, which also becomes salient in qualitative operator inspection (e.g. in \cref{fig:m}).
Similarly, the SCF assesses the degree of spatial resolution loss in a quantitative fashion.
By comparing subject-wise and session-wise variance, the operator can ascertain how much a registration pipeline is potentially overfitting.
These metrics are of significant relevance to improving mouse brain MRI workflows, and could themselves be further optimized (e.g. by developing percentile selection heuristics based on a priori documented data distortions for VCF).

Global statistical power is not (at least in the range of workflows at hand) sensitive to registration.
It is thus not a reliable metric for optimization, though regrettably, it may be the most prevalently used if results are only inspected at a higher level --- and thus lead to analysis biasing in favour of a hypothesis.
This is demonstrated by the positive interaction effect of the Legacy workflow level and the CBV contrast level seen in \cref{fig:mscv}: in this particular case, optimizing for statistical power alone may give a misleading indication.
We do not discount this measure entirely, however, as it is strongly sensitive to workflow parameter variations which we have excluded for the sake of brevity in this comparison (such as registration interpolation method).

Overall we suggest that a VCF, SCF and Variance based comparison, coupled with visual inspection of a small number of omnibus statistic maps is a feasible and sufficient tool for benchmarking workflows, with MS usable as an additional sanity check.
We recommend reuse of the data herein presented for workflow benchmarking, as it includes (a) multiple sources of variation (contrast, session, subjects), (b) functional activity with broad coverage but spatially distinct features, and (c) significant distortions due to implant properties --- which are appropriate for testing the workflow robustness.
Owing to the RepSeP-compilant executable source code, which reproduces the statistics and figures in this document, our processing and data analysis is not only is fully transparent, but also reusable with further data and further workflows.

\subsection{Conclusion}

The SAMRI Generic workflow and Generic template presented in this article constitute a notable leap from the prevailing ad hoc paradigms of mouse brain imaging analysis.
This is attested by an in-depth multivariate comparison of this novel design with a thoroughly documented Legacy pipeline representing alternative practices.
For workflow comparison, we introduced metrics that can be used beyond the scope of this work for registration Quality Control.
The optimized registration parameters of our workflow are accessible in the source code and transferable to any other workflows making use of the ANTs package.
The software engineering choices in both the workflow and this article's source code empower users to better verify, understand, remix, and reuse our work.
Overall we believe that the insights summarized and technologies showcased herein will have a significant role in profoundly improving computational mouse brain imaging methodology.

\section{Extra}

%The problem with evaluating preprocessing pipelines is, that we lack ground truth.
%Therefore any approach learning is potentially prone to overfitting.
%with our novel workflow or with the Here we derive two novel metrics for assessing the quality of the registration, which were not used by engineering the new workflow nor used in the optimisation procedure. 

It is significantly more time-consuming and primarily relevant as a demonstration rather than an efficient way to test, analyze, and incrementally improve preprocessing workflows.

The templates used for mouse brain MRI are highly heterogeneous, limiting data integration potential.
Affine “removal” comes to the strong detriment to the resulting neuroimaging data, which persists in all downstream statistics produced from such data.
This is due to the fact that visual representation unavoidably requires an affine transformation (to turn data points into volumes in space), and the deletion of this information makes the spatial representation more rather than less ambiguous.
One common issue which may arise is that software (rightfully) attempts to recreate the affine information.
Such data recovery behaviour is not determined by the NIfTI (or any other) standard, and as such is unpredictable.
An illustration of this issue ls given by the comparison of coordinates in the MRIcroGL and SAMRI (internally using NiLearn) plots of the legacy pipeline results.

While a population template may require less deformation, it only permits within-study comparability, and requires additional interpolation in order to allow region of interest (ROI) delineation or inter-study comparison.
While it is beyond the scope or intent of this study to single out and individually investigate articles by our colleagues, we offer a comparison from our own data and pipelines, illustrating the benefits of context awareness and specific optimizations in animal MRI.
Destructive hacks in particular (such as affine transformation deletion or scaling) preclude both the usage of preprocessed data and of the preprocessing workflow itself in size-sensitive applications, which may include a plethora of diagnostic or preoperative imaging scenarios (e.g. !!!cite StereotaXYZ!!!).

Human MRI research has produced numerous registration toolkits and associated workflow implementations, predominantly accessed via high-level interfaces contain hard-coded parameters optimized for specific human MRI use cases.
Animal MRI commonly makes use of these high-level interfaces, and implements additional hacks to mitigate the nonhuman idiosyncracies of the species being imaged --- instead of optimizing the workflow for the data at hand.
Quality control is commonly performed by operator inspection, making it infrequent, biased, slow, and unreproducible.
In this paper we present a novel workflow using the full flexibility of low-level functions from one of the most popular neuroimaging registration toolkits, and provide an optimized set of parameters for small animal imaging.
Additionally, we present a quality control (QC) workflow, which can automatically assess the registration quality of processed datasets.
We showcase the capabilities of both workflow, by comparing our current registration performance with that of a legacy registration workflow (containing multiple popular hacks - which we specifically critique).

\subsection{Variance Analysis}

While the absolute value of image similarity metrics cannot be relied upon for QC, the variance structure of similarity metrics in longitudinal datasets can.
The rationale of analysing similarity metric variance is that in healthy adult mice, there is no significant change in brain structure over time, but there are significant differences in individual anatomy.
This proposition is further strengthened in our dataset, as each animal has a slightly different but temporally stable implant placement. 
The corollary of this statement in terms of value distributions, is that the more variance is reduced across sessions without being reduced across subjects, the more stable a workflow is and the less likely it is for its performance to be derived from over-fitting.


We use this metric to optimize our registration parameters, as it is uniquely suites to our dataset (as opposed to to other implant-less data sets), and allows us to leverage it to a maximum use for the community, while keeping other less intricate and unrelated metrics usable for transferable QC for external datasets.


\[ CC(x) = \frac{\sum_i{(x)}}{\sum_i{(x)}} \]

%\py{pytex_fig('scripts/registration_qc.py', conf='article/varplot.conf', label='varplot', caption='Variance for different preprocessing pipelines')}

To assess the quality of the pipeline we evaluated the registration performance for different metrices (Crosscorrelation (CC), Mutual Information (MI), Mean Squared Difference (MSQ)) for individual sessions and subjects on a representatitve dataset.
We define an assessment for registration quality based on the assumption, that for increased registration quality the variance of a similariy measure between the subject and the template should converge towards 0.
This definition is based on the assumption that biological deformations of the brain across sessions should be negligible (Ref???!?!).
Hence we calculate the variance over different similiarity metrics for each subject across sessions.
We average for each workflow the results across subjects.
We find that our new preprocessing pipeline has significantly less variance than the legacy workflow, while the optimised pipeline has even further decreased variance.

\subsection{Functional Feature Mapping}

Distinct functional features, if available, can be tracked to provide QC supplementing the analysis of overall statistical power (as estimated by MS).
This method is even more restricted in general applicability than the global functional analysis, as it requires an a priori known feature which is easily distinguished via statistical estimates from its immediate surroundings.
The data set at hand, however, presents such a feature in the form of the DR, which is targeted for stimulation and know a priori (cite Joanes!!!) to present activation, contrasting with concomitant deactivation in other areas of the brain. 
To ensure sensitivity to the directionality of the response we analyze the per-animal mean t-statistic in the DR.

\begin{sansmath}
\py{pytex_subfigs(
        [
                {'script':'scripts/fdrtc_violin.py', 'label':'facv','conf':'article/1col.conf', 'options_pre':'{.48\\textwidth}',
                        'caption':'Comparison across workflows and functional contrasts, considering only matching template-workflow combinations.'
                        ,},
                {'script':'scripts/fdrt_violin.py', 'label':'fav', 'conf':'article/1col.conf', 'options_pre':'{.48\\textwidth}',
			'caption':'Comparison across workflows and target templates, considering only CBV functional contrasts.'
                        ,},
                ],
        caption='Mean per-animal t-value in Dorsal Raphe. Coloured patch width estimates distribution density, while continuous markers indicate the sample mean and dashed markers indicate the inner quartiles.',
        label='fig:fa',)}
\end{sansmath}

A salient feature seen in this comparison, is that With respect to the data break-up by contrast (from \cref{fig:facv}), we see a notable positive main effect for the contrast variable
(Dorsal Raphe t-statistic mean of
\py{boilerplate.corecomparison_factorci('Contrast[T.CBV]', df_path='data/functional_t.csv', dependent_variable='Mean DR t')}
).
This encourages a qualitative analysis of the statistical distribution maps, which we present in \cref{fig:m}.

As seen in \cref{fig:fa}, we note that functional score is sensitive to neither
the processing workflow
(\py{boilerplate.fstatistic('Processing', dependent_variable='Mean DR t', df_path='data/functional_t.csv', condensed=True)}),
nor the template
(\py{boilerplate.fstatistic('Template', dependent_variable='Mean DR t', df_path='data/functional_t.csv', condensed=True)}),
nor the interaction thereof
(\py{boilerplate.fstatistic('Processing:Template', dependent_variable='Mean DR t', df_path='data/functional_t.csv', condensed=True)}).

Testing the core hypothesis of the comparison ---
whether the Generic SAMRI workflow (with the Generic template) performs significantly different than the Legacy workflow (with the Legacy template) ---
we note that it does not
(two-tailed p-value of
\py{pytex_printonly('scripts/fdrs_t.py')}).
As before, we note that there is a variance increase in all conditions for the Legacy processing workflow
(\py{boilerplate.varianceratio(df_path='data/functional_t.csv', template='Legacy', dependent_variable='Mean DR t')}-fold
given the Legacy template, and
\py{boilerplate.varianceratio(template='Generic')}-fold
given the Generic template).

%\bibliographystyle{unsrtnat}
%\setcitestyle{square,numbers}
\printbibliography

\end{document}

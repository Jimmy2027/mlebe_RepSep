\documentclass[10pt,a4paper,twocolumn,german]{article}

\title{An Optimized Registration Pipeline and Standard Space for Small Animal Brain Imaging}
\author{
	\authorstyle{Horea-Ioan Ioanas\textsuperscript{1}}
	\authorstyle{Markus Marks\textsuperscript{2}}
	\authorstyle{Mehmet Fatih Yanik \textsuperscript{2}}
	\authorstyle{Markus Rudin\textsuperscript{1}}
	\newline
	\textsuperscript{1}\institution{Institute for Biomedical Engineering, ETH and University of Zurich}\\
	\textsuperscript{2}\institution{Institute of Neuroinformatics, ETH and University of Zurich}
}
\date{}
\input{article_style.tex}

\begin{document}
\maketitle
\thispagestyle{firstpage}
\noindent
\textsf{%
\textbf{Abstract ---}
	Given the need for inter-subject and inter-study comparability, spatial map registration to a standardized space is an indispensable part of modern Magnetic Resonance Imaging (MRI).
Mouse MRI workflows commonly utilize high-level interfaces optimized for human data --- and adapt the data to the interface rather than vice-versa.
Quality control (QC) is commonly performed by interactive operator inspection, making it infrequent, open to bias, slow, and unreproducible.
In this paper we present a novel registration workflow accessible via both Bash and Python, which uses the full flexibility of low-level interfaces from one of the most popular normalization toolkits (ANTs).
We provide an optimized set of parameters for mouse brain registration, and propose a standard space suited to harmonize mouse brain data across modalities.
Additionally, we present QC workflows, which can automatically assess the registration quality of current as well as past processed datasets.
We showcase the capabilities of this novel workflow compared to a legacy workflow (representative of common practices --- which we detail and comment).

}

\section{Background}
In order to make meaningful comparisons across subjects inside a study, it is imperative that the images lie in a standard reference frame.
%Voxel-based population studies of either functional or structural variables depend on mapping to a template space.
%The common coordinate system enables a statistical evaluation of the likelihood of consistent activation across a group or, in other contexts, the differences in anatomy between two groups.
Because of positioning imprecision and anatomical animal variations, this is not the case for the original MR acquired images.
To solve this issue, the images need to be projected into the reference frame via registration \cite{maintz_overview_nodate, sotiras_deformable_2013}.
As reported by Ioanas et al. \cite{ioanas_optimized_2019}, the general approach for mouse-brain image registration is to use high-level functions designed and optimized for human brain images.
This requires the mouse-data to be adapted to the processing function instead of vice-versa.
To provide contrast, they compare two workflows, a Legacy workflow that adapts the data to the processing functions and a Generic workflow, which is optimized to the data.
While the Legacy workflow expands voxel size and deletes orientation information of the affine matrix in order to fit human brain data, the Generic workflow uses functions provided by the ANTs package \cite{ants}, with spatial parameters adapted to the mouse brain.
In a quality control, it is shown that the Generic workflow improves volume conservation, smoothness conservation and provides a reduction in variance.
While the performance increase is considerable, registration quality can be improved further by computing the transformation solely on the brain volume to reduce disturbances induced by intensity variations outside the brain region.

\section{Methods}\label{sec:methods}
For the benchmarking of the two workflows, the same methods that are described in the original paper have been applied in this work.
A more detailed description can be found there.

\begin{sansmath}
\py{pytex_subfigs(
        [
                {'script':'scripts/classifier/uprepex.py', 'label':'exunprepro', 'conf':'article/1col.conf', 'options_pre':'{.48\\textwidth} \\vspace{-2em}',
                        'options_pre_caption':'\\vspace{0.1em}',
                        'options_post':'\\vspace{1em}',
                        'caption':'Example of an unpreprocessed slice.'
                        ,},
                {'script':'scripts/classifier/prepex.py', 'label':'exprepro', 'conf':'article/1col.conf', 'options_pre':'{.48\\textwidth} \\vspace{-2em}',
                        'options_pre_caption':'\\vspace{0.1em}',
                        'options_post':'\\vspace{1em}',
                        'caption':'Example of a preprocessed slice.'
                        ,},
                ],
        caption='\\textbf{The preprocessing removes the mask there, where the image-pixelvalues are 0.}\\
        Plots of the same image, superposed with the template mask, with and without preprocessing.
        ',
        label='fig:prepro_examples',
        )}
\end{sansmath}

\begin{sansmath}
\py{pytex_fig('scripts/classifier/plt_trainset.py',
        conf='article/2*8plot.conf',
        label='trainset',
        caption='
        Augmented samples from the Training set.
        ',
        multicol=True,
        )}
\end{sansmath}

\begin{sansmath}
\py{pytex_fig('scripts/classifier/show_blacklist.py',
        conf='article/2*8plot.conf',
        label='bl',
        caption='
                \\textbf{Slices where the mask includes too much outer-brain intensities are excluded from the data set.}\\
                Examples from the slices that were excluded from the data set. The mask is shown in blue, on top of the brain image.
                ',
        multicol=True,
        )}
\end{sansmath}

\subsection{Model}
As the architecture of the classifier, the U-Net from Ronneberger et al \cite{ronneberger_u-net:_2015} was chosen based on its high performance in the field of biomedical image segmentation.
This is a convolutional neural network that consists of a contracting path that captures context in addition to a symmetric expanding path that enables precise localization.
Localization in this context means that a class label is assigned to each pixel.
We used the attention gated U-Net implementation from Ozan Oktay et al. \cite{oktay_attention} for which the code is publicly available \cite{oktay_ozan-oktayattention-gated-networks_2020}.
Additionally to the original U-Net structure, their implementation has attention gates in the expanding part which weight the information coming from the symmetric counterpart.
The additional parameters in these attention gates allow the model to learn which region of the image is important for specific tasks and to suppress irrelevant regions.
In our use case this implementation helped reduce false positive classifications of high intensities, outside of the mouse brain region.

The model was trained using the Dice loss, which is computed from the Dice score.
It calculates the similarity of two binary samples X and Y with
\begin{equation}\label{eqDcoef}
D_{coef} = \frac {2|X\cap Y| + \epsilon}{|X|+|Y| + \epsilon}
\end{equation}
where a smoothing factor $\epsilon$ of 0.01 is used.

It is a quantity ranging from 0 to 1 that is to be maximized.
The loss is then calculated with $1-D_{coef}$.

Because many more pixels in the masks are 0 than 1, there is a class imbalance problem.
This is a problem because in this case a false positive gives a much higher loss than a false negative.
For example, predicting only black would give an acceptable loss, while predicting only white pixels would not.
Using the Dice coefficient as a loss function for training should make it invariant to this class imbalance problem as stated by Fausto Milletari et al. in \cite{milletari_v-net:_2016}.

\subsection{Data Set} \label{subsec:Data Set}
The data set consists of 3D MR images taken from an aggregation of three studies: irsabi \cite{irsabi_bidsdata}, opfvta \cite{ioanas_whole-brain_nodate}, drlfom \cite{ioanas_effects_nodate} and other unpublished data, acquired with similar parameters.

The irsabi data set consists of 102 scans coming from 11 adult animals, each scanned in up to 5 sessions with a 7T Bruker PharmaScan.
The sessions were repeated at 14 days intervals, each containing one anatomical (echo-time: 21ms, inter-echo spacing: 7ms, repetition time (TR): 2500ms) and two functional (CBV and BOLD with a flip angle of 60°) scans.
The functional scans were sampled at $\mathrm{\Delta x(\nu)=\SI{312.5}{\micro\meter}}$, $\mathrm{\Delta y(\phi)=\SI{281.25}{\micro\meter}}$, and $\mathrm{\Delta z(t)=\SI{650}{\micro\meter}}$ (slice thickness of \SI{500}{\micro\meter}).

The opfvta data set consists of \py{boilerplate.get_nmbrScans_from_dataselection('opfvta')} scans coming from \py{boilerplate.get_nmbrSubject_from_dataselection('opfvta')} adult animals, each scanned in up to \py{boilerplate.get_max_numbrSession_from_dataselection('opfvta')} sessions with a 7T Bruker PharmaScan.
The sessions were repeated at ??? days intervals, each containing one anatomical (echo-time: 30ms, inter-echo spacing: 10ms, repetition time (TR): 2950ms) and a functional (CBV with a flip angle of 60°) scan.
The functional scans were sampled at $\mathrm{\Delta x(\nu)=y(\phi)=\SI{75}{\micro\meter}}$ and a slice thickness of $\mathrm{\Delta z(t)=\SI{450}{\micro\meter}}$.

The drlfom data set consists of \py{boilerplate.get_nmbrScans_from_dataselection('drlfom')} scans coming from \py{boilerplate.get_nmbrSubject_from_dataselection('drlfom')} adult animals, each scanned in up to \py{boilerplate.get_max_numbrSession_from_dataselection('drlfom')} sessions with a 7T Bruker PharmaScan.
The sessions were repeated at ??? days intervals, each containing one anatomical (echo-time: 30ms, inter-echo spacing: 10ms, repetition time (TR): 2950ms) and a functional (CBV with a flip angle of 60°) scan.
The functional scans were sampled at $\mathrm{\Delta x(\nu)=y(\phi)=\SI{225}{\micro\meter}}$, and a slice thickness$\mathrm{\Delta z(t)=\SI{450}{\micro\meter}}$.

The measured animals were fitted with an optic fiber implant ($\mathrm{l=\SI{3.2}{\milli\meter} \ d=\SI{400}{\micro\meter}}$) targeting the Dorsal Raphe (DR) nucleus in the brain stem.
Using this dataset shows that the classifier is robust to these types of experiment setups.

Images from the irsabi study are only used for quality control of the registration and are thus unknown to the classifier.
It is the same dataset that was used to benchmark the Generic workflow in the original paper and thus allows for a better estimation of the general performance of our improved pipeline.

The images are transformed into a standard space using a template mask via SAMRI \cite{noauthor_ibt-fmi/samri_2019} and are thus defined in the same affine space.
SAMRI is a data analysis package of the ETH/UZH Institute for Biomedical Engineering.
It is equipped with an optimized registration workflow and standard geometric space for small animal brain imaging \cite{ioanas_optimized_2019}.

Because of variance in mouse brain anatomy and in the experiment setup, some of the transformed data do not overlap perfectly with the reference template.
To filter these images out, most of the incongruent volumes were removed manually from the data set.

For the registration of the images, a padding was needed to make the originally not affine space affine.
As a result, the 3D volumes present many zero-valued slices, some of them overlapping with the mask.

Since it is not wanted for the model to predict a mask on black slices, the mask is set to zero where the image is zero-valued.
This has also the advantage of bringing variance into the template.
Because some pixels representing the brain tissue are zero-valued, holes result from this operation.
To patch these, the function \textcolor{mg}{\texttt{$binary\_fill\_holes$}} from scipy.ndimage.morphology \cite{noauthor_multi-dimensional_nodate} is used.
An example of the preprocessing can be seen in \cref{fig:prepro_examples}.

In the coronal view, each slice of the transformed data is originally of shape (63, 48), matching the reference space resolution of \SI{200}{\micro\metre}.
It is then reshaped into \py{boilerplate.get_training_shape('tuple')} by first zero-padding the smaller dimension to the same size as the bigger one and then reshaping the image into \py{boilerplate.get_training_shape()} using the function \textcolor{mg}{\texttt{$cv2.resize$}} from the opencv python package \cite{noauthor_opencv-python_nodate}.

Finally, the images are normalized by first clipping them from the minimum to the \nth{99} percentile of the data to remove outliers and then divided by the maximum.

The data set is separated into Training, Validation and Test sets such that 90\% of the total data are used for training and validation while 10\% are used for testing.
This is done with the help of the function \textcolor{mg}{\texttt{$train\_test\_split$}} from the package sklearn.model$\_$selection \cite{scikit-learn}.
The Validation set is used for the optimization of hyperparameters while the Test set is used as a measure of extrapolation capability.
The irsabi data was additionally added to the test set.

\subsection{Data Augmentation} \label{Data Augmentation}

Because of diverse settings in the experiment setup, including animal manipulations causing artifacts, MR image quality can differ substantially between labs and even individual study populations.
To account for these variations, we apply an extensive set of transformations to our data.
This includes rotations of up to 20$^{\circ}$, a zoom range of -0.2 to +0.1, a random bias field added on the images and horizontal as well as vertical flips.
Additionally a gaussian noise is added to the images.

This not only increases the data set size but also makes it more representative of the general data distribution of mice brain MR images and results in a model with a better generalization capability.

\subsection{Training}
The model was trained on 3D volumes in the coronal view.
It was chosen over the axial one, because the shapes of the masks are much simpler in the coronal view and thus easier to learn for the network.

Additionally, the coronal view has the advantage of higher resolution as the MR images were recorded coronally.

\subsection{Masking}
To improve the SAMRI registration workflow, an additional node is implemented where the images are masked, such that only the brain region remains.
%To alleviate the task of the classifier, the image is first bias-corrected using the "\textcolor{mg}{\texttt{$N4BiasFieldCorrection$}}" function of the ANTs package, with spatial parameters used in the samri functions.
The image is first resampled into the resolution of the template space, which has a voxel size of $0.2\times 0.2 \times 0.2$.
This is done with the \textcolor{mg}{\texttt{$Resample$}} command from the FSL library which is an analysis tool for FMRI, MRI and DTI brain imaging data \cite{fsl}.
Then, the image is preprocessed using the operations described in \cref{subsec:Data Set}.
Since the classifier was trained to predict on images of shape \py{boilerplate.get_training_shape('tuple')}, the input needs to be reshaped.
The predictions of the model are reconstructed to a 3D mask via the command \textit{Nifit1Image} from the neuroimaging python package nibabel \cite{noauthor_neuroimaging_nodate}.
This is done using the same affine space as the input image.
The latter is then reshaped into the original shape inverting the preprocessing step, either with the opencv resize method or by cropping.
Additionally, the binary mask is resampled into its original affine space, before being multiplied with the brain image to extract the ROI.

\subsection{Metrics}

The VCF uses the 66\textsuperscript{th} voxel intensity percentile of the raw scan before any processing as definition of the brain volume.
The VCF is then obtained with \cref{eq:vcf}, where $v$ is the voxel volume in the original space, $v^\prime$ the voxel volume in the transformed space, $n$ the number of voxels in the original space, $m$ the number of voxels in the transformed space, $s$ a voxel value sampled from the vector $S$ containing all values in the original data, and $s^\prime$ a voxel value sampled from the transformed data.

\begin{equation} \label{eq:vcf}
        V\!C\!F
        = \frac{v^\prime\sum_{i=1}^m [s^\prime_i \geq P_{66}(S)]}{v\sum_{i=1}^n [s_i \geq P_{66}(S)]}
        = \frac{v^\prime\sum_{i=1}^m [s^\prime_i \geq P_{66}(S)]}{v \lceil0.66n\rceil}
\end{equation}

The SCF metric is based on the ratio of smoothness before and after processing.
It is obtained via \cref{eq:acf}, where $r$ is the distance of two amplitude distribution samples, $a$ is the relative weight of the Gaussian term in the model, $b$ is the width of the Gaussian and $c$ the decay of the mono-exponential term \cite{cox2017fmri}.

\begin{equation} \label{eq:acf}
        ACF(r)
        = a * e^{ -r^{2}/ (2 * b^{2}) } + (1 - a) + e^{-r/c}
\end{equation}

The for the MS relevant statistical power is obtained via the negative logarithm of first-level p-value maps.
Voxelwise statistical estimates for the probability that a time course could --- by chance alone --- be at least as well correlated with the stimulation regressor as the voxel time course measured are averaged via \cref{eq:ms}, where $n$ represents the number of statistical estimates in the scan, and $p$ is a p-value.

\begin{equation} \label{eq:ms}
        M\!S = \frac{\sum_{i=1}^n -log(p_i)}{n}
\end{equation}

\subsection{Statistics}

In the methods section, all statistics are presented with respect to the distributions of the absolute distances to 1, |1 - Metric|.
Based on a Likelihood Ratio Test, we chose models that do not examine the Workflow- Contrast interaction.
The full summaries of the analysis can be seen in tables \cref{table1},\cref{table2},\cref{table3} and \cref{table4}.

\section{Data and Code Availability}

%The data archive relevant for this article is freely available \cite{mlebe_bidsdata}, and automatically accessible via the Gentoo Linux package manager.
In addition to the workflow code \cite{mlebe, samri}, we openly release the re-executable source code \cite{mlebe_repsep} for all statistics and figures in this document.
The herein introduced novel method as well as the benchmarking are thus fully transparent and reusable for further data.
\section{Evaluation}

%The problem with evaluating preprocessing pipelines is, that we lack ground truth.
%Therefore any approach learning is potentially prone to overfitting.
We evaluate the quality of the registration both in terms of spatial features, as well as in terms of its repercussion on higher-level functional analysis.
%with our novel workflow or with the Here we derive two novel metrics for assessing the quality of the registration, which were not used by engineering the new workflow nor used in the optimisation procedure. 

A main challenge of QC with regard to spatial features is that a perfect remapping is undefined.
Similarity metrics are ill-fitted for QC because they are used internally by registration functions, whose main feature it is, that they maximize them.
Indeed an extreme maximization, especially via nonlinear transformations, results in a distortion of the image, which should be penalized in QC, but in light of image similarity scores, is represented as better performance.
Additionally, similarity metrics are not independent, so this issue cannot be circumvented by maximizing a subset of metrics and performing QC in light of the remainder.
We thus develop three alternative evaluation metrics: volume conservation, functional analysis, and variance analysis.

\subsection{Volume Conservation}

We have developed a simple, fast, and widely applicable metric to measure distortion introduced by preprocessing workflows.
Volume conservation is based on the assumption that the total volume of the scanned segment of the brain should remain approximately identical after registration.
A volume increase may indicate that the brain was stretched to fill in template brain space not covered by the scan, while a volume decrease might indicate that non-brain voxels were introduced into the template brain space.

Reference brain volume is estimated as the 66\textsuperscript{th} percentile of the unregistered scan.
The arbitrary unit equivalent of this percentile threshold is recorded for each scan and applied to all registration workflow results for that scan, to obtain transformed brain volume estimates.
In order to mitigate possible differences arising from template size, we perform a multivariate analysis of both template and analysis.
In order to best analyze volume conservation, a Volume Change Factor (VCF) is computed for each processed scan, whereby volume conservation is highest for a VCF equal to 1.

% This metric provides distortion checking rather than goodness-of-fit qantification, which is, as previously described, difficult to do in a lean automated fashion.

\begin{sansmath}
\py{pytex_subfigs(
        [
                {'script':'scripts/vc_violin.py', 'label':'vcv', 'conf':'article/1col.conf', 'options_pre':'{.48\\textwidth}',
			'caption':'Comparison across workflows and target templates, considering both BOLD and CBV both functional contrasts.'
                        ,},
                {'script':'scripts/vcc_violin.py', 'label':'vccv','conf':'article/1col.conf', 'options_pre':'{.48\\textwidth}',
                        'caption':'Comparison across workflows and functional contrasts, considering only matching template-workflow combinations.'
                        ,},
                ],
        caption='Volume change relative to the original scan volume. Coloured patch width estimates distribution density, while continuous markers indicate the sample mean and dashed markers indicate the inner quartiles.',
        label='fig:vc',)}
\end{sansmath}

As seen in \cref{fig:vcv}, we note that the Volume Change Factor (VCF) metric is sensitive to both
the processing workflow (\py{boilerplate.fstatistic('Processing', condensed=True)}),
the template (\py{boilerplate.fstatistic('Template', condensed=True)}),
and interactions thereof (\py{boilerplate.fstatistic('Processing:Template', condensed=True)}).

Testing the core hypothesis of the comparison ---
whether the Generic SAMRI workflow (with the Generic template) performs significantly different than the Legacy workflow (with the Legacy template) ---
we note that it does
(two-tailed p-value of \py{pytex_printonly('scripts/vc_t.py')}).
Additionally we note a root mean squared error ratio strongly favouring the Generic workflow
($\mathrm{RMSE_{L}/RMSE_{G}\simeq} \py{pytex_printonly('scripts/vc_rmser.py')}$).

Descriptively, we observe that the effect with the greatest magnitude is that of the template variable, with its Legacy level introducing a notable volume loss
(VCF of \py{boilerplate.vc_factorci('Template[T.Legacy]')}).
Further, we note that there is a variance increase in all conditions for the Legacy processing workflow
(\py{boilerplate.varianceratio(template='Legacy')}-fold given the Legacy template, and \py{boilerplate.varianceratio(template='Generic')}-fold given the Generic template).

With respect to the data break-up by contrast (from \cref{fig:vccv}), we see no notable main effect for the contrast variable
(VCF of \py{boilerplate.vcc_factorci('Contrast[T.CBV]')}).
We do, however, report a notable effect for the contrast-template interaction, with the Legacy workflow and CBV contrast interaction level introducing a volume loss
(VCF of \py{boilerplate.vcc_factorci('Processing[T.Legacy]:Contrast[T.CBV]')}).

\subsection{Functional Analysis}

Functional analysis, successfully circumvents the issue, as the metric being maximized in the registration process is not the same metric used for QC.
This method is however primarily suited to demonstrate workflow relevance to higher-level applications.

\subsection{Variance}

\[ CC(x) = \frac{\sum_i{(x)}}{\sum_i{(x)}} \]

%\py{pytex_fig('scripts/registration_qc.py', conf='article/varplot.conf', label='varplot', caption='Variance for different preprocessing pipelines')}

To assess the quality of the pipeline we evaluated the registration performance for different metrices (Crosscorrelation (CC), Mutual Information (MI), Mean Squared Difference (MSQ)) for individual sessions and subjects on a representatitve dataset.  We define an assessment for registration quality based on the assumption, that for increased registration quality the variance of a similariy measure between the subject and the template should converge towards 0. This definition is based on the assumption that biological deformations of the brain across sessions should be negligible (Ref???!?!). Hence we calculate the variance over different similiarity metrics for each subject across sessions. We average for each workflow the results across subjects. We find that our new preprocessing pipeline has significantly less variance than the legacy workflow, while the optimised pipeline has even further decreased variance.

\section{Optimization}

Next, we decided to optimize our workflow further by using the previously established metrics to define an objective function. We employ methods from derivative free optimization, since some of the workflow inputs are not differentiable.
\begin{equation}
  F  = Volume + Var
\end{equation}

\section{3D Plotting}
	In this section we demonstrate 3D plotting performance for this document class.

	Matplotlib's 3D plots are associated with a number of intricacies (such as axes.facecolor specifying the background of the 2D output figure and not the 3D grid --- which is why they are particularly interesting. 
	
	\py{pytex_fig('scripts/3dplot.py', conf='article/3dplot.conf', label='3dplot', caption='A 3D plot.')}

\section{Other Plots}
	Here we demonstrate more generic plotting capabilities.

	\py{pytex_fig('scripts/bsc_percentage.py', conf='article/bsc_percentage.conf', label='bsc_percentage', caption='Percentage of Bachelor’s degrees conferred to women in the U.S.A. by major (1970-2011).', multicol=True)}

\subsection{Extra}
The templates used for mouse brain MRI are highly heterogeneous, limiting data integration potential.
Affine “removal” comes to the strong detriment to the resulting neuroimaging data, which persists in all downstream statistics produced from such data.
This is due to the fact that visual representation unavoidably requires an affine transformation (to turn data points into volumes in space), and the deletion of this information makes the spatial representation more rather than less ambiguous.
One common issue which may arise is that software (rightfully) attempts to recreate the affine information.
Such data recovery behaviour is not determined by the NIfTI (or any other) standard, and as such is unpredictable.
An illustration of this issue ls given by the comparison of coordinates in the MRIcroGL and SAMRI (internally using NiLearn) plots of the legacy pipeline results.

While a population template may require less deformation, it only permits within-study comparability, and requires additional interpolation in order to allow region of interest (ROI) delineation or inter-study comparison.
While it is beyond the scope or intent of this study to single out and individually investigate articles by our colleagues, we offer a comparison from our own data and pipelines, illustrating the benefits of context awareness and specific optimizations in animal MRI.
Destructive hacks in particular (such as affine transformation deletion or scaling) preclude both the usage of preprocessed data and of the preprocessing workflow itself in size-sensitive applications, which may include a plethora of diagnostic or preoperative imaging scenarios (e.g. !!!cite StereotaXYZ!!!).

Human MRI research has produced numerous registration toolkits and associated workflow implementations, predominantly accessed via high-level interfaces contain hard-coded parameters optimized for specific human MRI use cases.
Animal MRI commonly makes use of these high-level interfaces, and implements additional hacks to mitigate the nonhuman idiosyncracies of the species being imaged --- instead of optimizing the workflow for the data at hand.
Quality control is commonly performed by operator inspection, making it infrequent, biased, slow, and unreproducible.
In this paper we present a novel workflow using the full flexibility of low-level functions from one of the most popular neuroimaging registration toolkits, and provide an optimized set of parameters for small animal imaging.
Additionally, we present a quality control (QC) workflow, which can automatically assess the registration quality of processed datasets.
We showcase the capabilities of both workflow, by comparing our current registration performance with that of a legacy registration workflow (containing multiple popular hacks - which we specifically critique).



\bibliographystyle{unsrtnat}
%\setcitestyle{square,numbers}
\bibliography{./bib}

\end{document}

% Preamble
\documentclass[11pt, english]{article}
\usepackage[backend=bibtex,style=authoryear,natbib=true]{biblatex}
\usepackage{cleveref}
\usepackage{sansmath}
\addbibresource{bib.bib}

% Single-session PythonTeX codeblocks
\newcounter{pysessioncounter}
\newcommand{\sessionpy}{%
    \edef\sessionpysession{session\arabic{pysessioncounter}}%
    \stepcounter{pysessioncounter}%
    \expandafter\py\expandafter[\sessionpysession]}

% Pythontex
\usepackage[autoprint=false, gobble=auto, pyfuture=all]{pythontex} %create figures on-line directly from python!
\usepackage{pgf} %we need this backend

\input{pythontex/functions.py}
\begin{pythontexcustomcode}[begin]{py}
    DOC_STYLE = 'article/main.conf'
    pytex.add_dependencies(DOC_STYLE,
    )
\end{pythontexcustomcode}



\usepackage{makecell}
\usepackage{siunitx}

\sisetup{detect-all} % will use the current font for typesetting
\newcommand\SIci[5]{\SI{#1}{#2}, {#3}CI: \SIrange{#4}{#5}{#2}}



\title{Machine Learning Enabled Brain Segmentation for Small Animal Image Registration}
\date{}

% Document
\begin{document}
    \maketitle


    \section{Introduction}
    Registration is needed to compare and measure imaging data across populations with subject-wise variation.
    Current methods for neuroimaging rely on full image processing, with high varying intensities outside the brain region of interest that interfere with registration.
    Applying the processing to a masked image improves the quality of the latter.
    We present a deep learning framework for segmentation of brain tissue in functional and structural MR images that significantly improves the quality of registration in small animal brain imaging.


    \section{Methods}
    We apply a machine learning auto-encoder type model as preprocessing to a registration workflow in order to generate a mask of the brain region.
    This mask is used to focus computation on the brain region during registration.
    As model architecture, we use the U-Net \citep{ronneberger_u-net:_2015} implementation from \citet{oktay_ozan-oktayattention-gated-networks_2020}.
    The model was trained on 3D MR images taken from an aggregation of multiple studies, namely opfvta \citep{ioanas_whole-brain_nodate}, drlfom \citep{imperfect_datasets} and other unpublished data, acquired with similar parameters.
    We compare our extended registration workflow to the SAMRI workflow \citep{samri} on data from the irsabi study
    \citep{irsabi_bidsdata}.


    \section{Results}
    In comparison to the SAMRI registration workflow, which does not focus computation on the brain region, our workflow exceeds in multiple metrics.
    The Volume Conservation Factor (VCF) \citep{ioanas_optimized_2019} which measures the registration induced deformation of the scanned brain, by computing the ratio of the brain volume before and after preprocessing, favours our workflow with a root mean squared error ratio of ($\mathrm{RMSE_{M}/RMSE_{G}\simeq} 0.66$).
    We also note that our workflow results in a inter-sample VCF variance decrease (0.44-fold).
    Evaluating the Smoothness Conservation Factor (SCF) \citep{ioanas_optimized_2019}, which expresses the ratio between smoothness of the registered image in comparison to the original, yields that the improvement of the VCF does not come at the cost of a reduced SCF: $\mathrm{RMSE_{M}/RMSE_{G}\simeq} 0.93$.
    \Cref{fig:regComp} shows a qualitative comparison between the registration quality of the two workflows.


    \begin{sansmath}
        \py{pytex_fig('scripts/classifier/plt_registration_comparison.py',
            conf='article/4*2.conf',
            label='regComp',
            caption='
            \\textbf{The Masked workflow prevents the shifting of outer-brain region voxels into the template-brain
            region (in blue).} Comparison of slices from 3 different volumes, registered with the Generic (first row)
            and the Masked (second row) workflow.',
            )}
    \end{sansmath}


    \section{Conclusion}
    We present a brain labeling classifier, that when used as a ROI extraction in an extention of the SAMRI Generic registration workflow, significantly improves the quality of the latter.
    Our extended workflow offers several advantages summarized by established metrics.
    The easily accessible registration parameters of the SAMRI Generic Workflow as well as the open source code to the classifier training functions make the pipeline transferable to any other imaging applications.
%    Visual inspection of registration quality reveals that the classifier successfully reduces the shifting of outer brain
%    region voxels into the template space.




    \clearpage
    \printbibliography
\end{document}

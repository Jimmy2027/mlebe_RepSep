% Preamble
\documentclass[11pt, english]{article}
\usepackage[backend=bibtex,style=authoryear,natbib=true]{biblatex}
\addbibresource{bib.bib}

\usepackage{makecell}
\usepackage{siunitx}

\sisetup{detect-all} % will use the current font for typesetting
\newcommand\SIci[5]{\SI{#1}{#2}, {#3}CI: \SIrange{#4}{#5}{#2}}



% Document
\begin{document}
    \section{Introduction}
    Cross-subject and cross-study comparability of preclinical imaging data, whole-brain imaging data in particular, is contingent on the quality of registration to a standard reference space.
    Current methods for neuroimaging rely on full image processing, with high varying intensities outside the brain region of interest that interfere with registration.
    Applying the processing to a masked image improves the quality of the latter.
    Here we present a deep learning enabled framework for segmentation of brain tissue in functional and structural MR images that when included in a small animal brain imaging workflow significantly improves the quality of the latter.


    \section{Methods}
    We apply a machine learning auto-encoder type model as preprocessing to a registration workflow in order to generate a mask of the brain region.
    This mask is used to focus computation on the brain region during registration.
    As model architecture, we use the U-Net \citep{ronneberger_u-net:_2015} implementation from \citet{oktay_ozan-oktayattention-gated-networks_2020}.
    The model was trained on 3D MR images taken from an aggregation of multiple studies, namely opfvta \citep{ioanas_whole-brain_nodate}, drlfom \citep{imperfect_datasets} and other unpublished data, acquired with similar parameters.
    We compare our extended registration workflow to the SAMRI workflow \citep{samri} on data from the irsabi study \citep{irsabi_bidsdata}.

    \clearpage
    \printbibliography
\end{document}
\section{Background}

In order to make any generalizable statements regarding brain function and organization, an equivalence between brain areas across individuals needs to be established.
This is most commonly done by spatial (rigid, affine, and non-linear) transformation of all brain maps in a study to a population or atlas template.
This proess, called registration, is consequently performed as part of any neuroimaging workflow attempting to produce spatially resolved and generalizable results.

The computations required for registration are commonly performed at the very onset of the preprocessing workflow (possibly after slice-timing correction),
though --- depending on the workflow --- the actual image manipulation may only take place once inter-subject comparison is needed, or at the very end for statistic map publishing !!!Cite paper which does this and says that it minimizes interpolation!!!
As a consequence of this marginal position of the processing step, and its general independence from the hypothesis and experimental design, registration is often relegated to default values and exempt from rigorous quality control (QC).
Human brain imaging specifically (and uniquely) benefits from high-level registration functions (shipped by most software packages providing registration functionality) optimized for the size and spatial features of the human brain --- thus mitigating the first of the aforementioned issues.
The latter issues, however --- a lack of accessible QC --- remains notable for brain registration in all species.

In small animal imaging (of which we are particularly focusing on the mouse) registration is currently most expediently and frequently performed by implementing a series of steps which serve to adjust the nature of small animal brain images to fit the parameters optimized for human brain processing.
The first such step consists of adjusting the voxel dimensions (as recorded in the NIfTI header !!!cite NIfTI!!!) so as to let the small animal data represent a volume corresponding to what high-level human-optimized registration functions expect (commonly, in the case of mice, this constitutes a 10-fold increase in each dimension).	
A second step consists of re-engineering the masking step (whereby the brain is isolated from the skull); the common human-optimized functions doing this cannot be as easily adapted via a metadata modification, since it is the very shape of the mouse brain that precludes human priors from working correctly.
While more intricate approaches --- such as adapting specific function parameters and priors to animal data \cite{rbet} --- have been successfully pursued, simpler solutions remain popular, and may consist in applying an empirically determined threshold prior to usage of a human brain masking function, or foregoing this step altogether.


The registration process consists of the computation of a transformation matrix (which can have multiple layers of complexity, e.g. combining a transformation from the functional to the structural scan with a transformation from the structural scan to a general template) and the application of the transformation matrix in s process called warping.

	\subsection{Extra}
	While a population template may require less deformation, it only permits within-study comparability, and requires additional interpolation in order to allow region of interest (ROI) delineation or inter-study comparison.
	While it is beyond the scope or intent of this study to single out and individually investigate articles by our colleagues, we offer a comparison from our own data and pipelines, illustrating the benefits of context awareness and specific optimizations in animal MRI.
	Destructive hacks in particular (such as affine transformation deletion or scaling) preclude both the usage of preprocessed data and of the preprocessing workflow itself in size-sensitive applications, which may include a plethora of diagnostic or preoperative imaging scenarios (e.g. !!!cite StereotaXYZ!!!).
